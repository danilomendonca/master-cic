\chapter{Related Work}\label{ch:related_work}

%\section{Goal-oriented modelling and analysis}

\section{Contextual Goal Model}

The Contextual Goal Model (CGM)~\cite{Ali:2010} proposes the contextualization of the intentional elements and relations in a goal model. From the activation of a root goal to the adoptability of tasks, CGM define different context implication that may affect the problem tackled by a system, the validity and the quality of possible solutions. CGM uses a graphical notation to associate context to their affected elements. In specific, context can be associated to root goals (activation), OR-decompositions (adoptability), means-end (adoptability), dependencies (activation), AND-decomposition (activation), contribution to softgoals (quality). 

%In CGM, a context is specified by a formula of world predicates. Predicates can be either facts, if they can be verified by the actor, or statements, if the opposite. 

The main benefits of the CGM are twofold. First, CGM enriches the the conventional goal model with the notation for the contextualization of intentional elements and relations. Second, it provides the modelling constructs to analyse and discover relevant information the system needs to capture in order to verify if a context applies, i.e., the rationale for context monitoring. This last contribution is useful for runtime monitoring and analysis of a self-adaptive system reflecting stakeholder's rationale and the environment in which the system operates~\cite{Ali:2010}. 

CGM provides a more realistic and precise contribution analysis contextualized by environment conditions, but does not change the nature of the GORE contribution analysis. In contrast to CGM, our work empathizes the formal verification of non-functional requirements in place of the more subjective contribution analysis based on the direct evaluation of the forward impact between alternatives and softgoals. However, our work has benefited from the CGM conceptual model and, to answer our RQ3, we evaluate the feasibility of a goal-oriented dependability analysis that preserves the context effects from a CGM over the requirements to meet and the adoptable means in the corresponding probabilistic verification model. 

%the main contribution of our work is the goal-oriented probabilistic verification of reliability related metrics that needs a more robust and less biased approach instead of the existing contribution analysis that is based on the direct evaluation of the forward impact between goals/tasks and softgoals. Our work has also benefited from the CGM conceptual model and to answer our RQ3 we evaluate the feasibility of a goal-oriented dependability analysis that preserves the context effects from a CGM in the corresponding probabilistic verification model. 

%, i.e., that includes different context effects in the probabilistic model to contextually estimate the probability of different system goals in being fulfilled as part of a more precise analysis.

%values of required non-functional attributes of the system and provide a reliable decision criteria for the selection of concurring alternatives in the goal model given different contexts.

\section{Runtime Goal Model}\label{sec:behaviour_specification}

Despite the use of goal models in the support of runtime monitoring and adaptation in many works, Dalpiaz et al. argued that these proposals are `using design artefacts for purposes they are not meant to, i.e., for reasoning about runtime system behaviour'. As such, they proposed a conceptual distinction between the static goal model, named Design Goal Models (DGM), and the Runtime Goal Model (RGM) that extend DGM with `additional state, behavioural and historical information about the fulfilment of goals'~\cite{Dalpiaz:2013}.

The main purpose of the RGM approach is to provide a behaviour specification for the fulfilment of goals and the execution of tasks. RGM defines a class model, while the Instance Goal Model (IGM) captures instance states of monitored goals and tasks that must conform to their class model, the RGM.  If an IGM violates its RGM, then a corrective action is expected to take place.

The IGM representation is built from algorithms that parses the execution traces of the instrumented implementation of a running system. The contribution, however, is restricted to the runtime regex and to the IGM algorithms. Other research questions concerning the percentage of success/failures for a given goal with or without temporal frames is not yet addressed by the framework~\cite{Dalpiaz:2013}.

Our proposed goal-oriented dependability analysis has benefited from the RGM as the PMC technique requires a system behaviour specification and the RGM provides a high-level description of a system behaviour. In our work, RGM leaf-tasks are mapped to a probabilistic verification model in PRISM language preserving its behaviour semantics. Our proposal aims to answer qualitative and quantitative questions about the time-unbounded success/failure probability in fulfilling different system goals. Time-bounded verification should be explored in future works.

%with the use of the runtime regular expression to have a behaviour specification and generate the probabilistic model for our goal-oriented reliability analysis. Also, IGM may be useful for collecting individual reliability metrics of individual tasks required by the PMC technique.

%In this section, further details about the runtime regex syntax and semantic will be explained as they are a central part of this proposal. 

Table~\ref{tab:RGM_REGEX} provides a textual description of each RGM rule and the corresponding meaning in terms of what behaviour it specifies and also an example from the MPERS runtime goal model of Figure~\ref{fig:MPERS_LR}. A formal and detailed description can be found in the RGM reference publication~\cite{Dalpiaz:2013}.

% Please add the following required packages to your document preamble:
% \usepackage{booktabs}
\begin{table}[h]
{\renewcommand{\arraystretch}{1.5}
\begin{tabularx}{\textwidth}{@{}l|X|X@{}}
\toprule
\textbf{Expression} & \textbf{Meaning}                                                                   & \textbf{Example (MPERS)} \\ 
E1;E2               & A goal/task E1 must be fulfilled/executed before E2.                               & $G1\textbf{;}G2\textbf{;}G3\textbf{;}G4$                 \\ 
E1\#E2              & Interleaved fulfillment/execution of goal/task E1 and E2.                          & $(G1;G2;G3;G4)\textbf{\#}G5$                   \\ 
E+n                 & Goal/task E must be fulfilled/executed $n$ times, with $n \textgreater 0$.             & $G22+2$                    \\
E\#n                & Interleaved fulfillment/execution of $n$ instances of E, with $n \textgreater 0$.      & -                        \\
E1|E2               & Fulfillment/execution of goal/task E1 is alternative with respect to E2. & $T23.0|T23.1$                \\ 
opt(E)              & Fulfillment/execution of goal/task E is not mandatory.                             & $opt(T17.2)$               \\  
try(E)?E1:E2        & If goal/task E succeeds, E1 must be fulfilled/executed; otherwise, E2.             & $try(T9.0)?skip:T9.1$        \\ 
skip                & No action. Useful for conditional ternary expressions involving two elements.      & $try(T9.0)?skip:T9.1$        \\  \bottomrule
\end{tabularx}
}
\caption{Description of RGM behaviour rules used by the proposal.}
\label{tab:RGM_REGEX}
\end{table}

% Our work does not cover instance and monitoring aspects and focuses on the \&V phase of RE to anticipate any violation and for the selection of the most appropriate concurrent alternative elicited for the system.

\section{Dependability Contextual Goal Model}

This work has been preceded by another proposal concerning goal-oriented requirements engineering, dependability analysis and dynamic contexts, namely the Dependability Contextual Goal Model (DCGM)~\cite{Mendonca:2014}. The contribution was focused on the context effect over both dependability requirements and dependability attributes based on declarative fuzzy logic rules. 

In DCGM, a failure classification scheme was used to classify the consequence level and domain of failures in achieving system goals. This process lead to the definition of dependability constraints that must be achieved by the means-end tasks used to fulfil leaf-goals in specific contexts of operation, i.e., to the specification of contextual dependability requirements (CDR). These requirements inherited the same form of the AwReq~\cite{Souza:2011}, but instead of being static, CDRs are associated to context conditions. Another DCGM characteristic is the contextual failure implication (CFI), which consisted of a dependability domain-specific contribution analysis supported by fuzzy logic to define IF-THEN rules between context conditions and the corresponding level of a dependability attribute of a given alternative, e.g., the reliability of a task in a context condition.

The main drawback of this proposal was the lack of scalability, as declarative rules must be provided for different goals, attributes and contexts, proving to be a time-consuming manual analysis task. A second problem was the subjectivity of the rules, as they were based in domain knowledge that was also used to shape membership functions. This problem, as much as in GORE contribution analysis, lead to the idea of coupling a more precise and reliable verification approach such as the PMC technique to a runtime goal model. Still, the idea of a failure classification and the specification of dependability requirements as non-functional constraints were kept in the current work.

\section{Awareness Requirements}

Souza et al.~\cite{Souza:2011} proposed the Awareness Requirements (AwReq) as a meta-requirements class in a goal model, i.e., AwReqs specify, among others, the success/failure rate for other requirements in the model, including goals, tasks, domain assumptions and even other AwReqs (*-meta-requirement). The objective is to enrich the original goal model with constraints for the system performance and to provide criteria for self-adaptation, as runtime AwReq violations should be addressed by corrective actions. AwReq are formalized by a temporal logic formula, namely the Object Constraints Logic with Temporal Message (OCLtm).

Despite its contribution to the specification of meta-requirements in goal models, AwReq do not provide an approach to analyse and validate its meta-requirements before system implementation or through system monitoring. Original GORE contribution analysis could be used to define the impact of a given alternative to some attribute or value composing one or more AwReqs, similarly to the DCGM~\cite{Mendonca:2014}. In contrast, our approach tackles the verification of domain-specific dependability meta-requirements through probabilistic model checking technique that can be performed at design-time to support alternative design decision or at runtime as part of a self-adaptation loop analysis.

\section{Formal TROPOS}

The idea behind the formalization of a goal model, as proposed by Formal TROPOS (FTROPOS)~\cite{Fuxman:2004}, is to provide a formal specification of sufficient and necessary conditions to create and achieve intentional elements like goals, tasks and dependencies in the a goal model and also invariants for each of these elements. In addition to this, new \textit{prior-to} links describe the temporal order of intentional elements. Also, cardinality constraints may be added to any link in the model. Finally, FTROPOS uses a first-order linear-time temporal logic as a specification language.

The nature of the verification proposed by FTROPOS is different from the PMC used by our work. FTROPOS aims to provide the information required for a consistency verification of the goal model. The verification is performed not only on the  conventional goal model syntax of intentional elements and relations, but also to domain specific information about how each element is created and fulfilled in time. Once the model starts to have more elements and relations, its consistency checking becomes non-trivial, justifying the use of a formal specification that can be automatically verified by a model checker tool. 
	
%build a probabilistic model representation of the goal model enriched by RGM behaviour specification and enable the

In contrast to FTROPOS, our proposal uses a probabilistic model checking technique for the verification of properties that depend on how activities in the model are organized in terms of time, cardinality, combination, non-determinism and the success probability of individual system tasks. While the FTROPOS maps its specification into a language treated by a symbolic model checker, our work maps a RGM specification into a probabilistic model treated by the PRISM probabilistic model checker. Despite the similarity in the behaviour specification, FTROPOS focuses on consistency verification and not on the quantitative analysis of the system performance and dependability.

\section{Reliability Analysis of Software Product Lines}



%For instance, if power consumption is to be checked, each activity has to be associated to a power consumption unit and the global consumption value is evaluated considering any non-determinism specified in the execution workflow. Thus, even if the Formal TROPOS language also provides behaviour specification in a goal model, it is tailored for consistency checking of requirements and not for the probabilistic verification of dependability and other non-functional requirements.

%\section{Software Product Lines}
%
%\subsection{Dependability Analysis in Ambient Assisted Living}

