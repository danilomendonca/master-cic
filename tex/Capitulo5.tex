\chapter{Proposal}\label{ch_proposal}

In the PMC technique adopted by this proposal, a behavioural specification, usually provided by UML activity and sequence diagrams, are manually converted to a probabilistic model in PRISM language. As a goal model goes from strategical root goal to operational leaf-goals, and each leaf-goal describes a desired state reachable by either a delegation to other actor or by a operational task, then a behaviour specification as proposed by the RGM may be seen as an activity diagram and be used to generate a probabilistic model in PRISM language. This allows the model checking of the corresponding goal model as a set of activities for which temporal and other behaviour aspects are specified by the runtime regex of the RGM.

\begin{itemize}

\item Making a different choice for underlying components: In some cases the replacement of a technical component for another of the same class can improve the quality of how they achieve their goal. For instance,
\medskip

\item Behaviour optimization: The quality may also depend on the pattern used for the activities execution. The specification of a different pattern may eliminate the non-functional violation. 
\medskip

\item Contextualizing the alternative: An alternative may only violate a NFR in specific contexts. In this case, different valid alternatives may be used according to the context of operation.
\medskip

\item Alternative disposal: If the alternative is in absolute violation or if its validity is restricted to contexts that have at least one other valid alternative, this branch can be eliminated from the model.

\end{itemize}

To evaluate the current proposal with the MPERS case study, we have used the a discrete-time Markov chain (DTMC) probabilistic model and focused on the verification of properties related to dependability, i.e., the reachability of the final success state of a set of goal model activities that represents: 

\begin{itemize}

\item if the set is composed of the minimum set of activities that satisfies the root goal: its global reliability;  

\item if the set is composed of the minimum set of activities that satisfies any lower-level goal: its local reliability.

\end{itemize}

Dependability analysis is used to provide information about different dependability attributes related to system failures. These metrics may be specified as non-functional requirements for isolated system functionalities or for the whole system. Instead of softgoals, we use meta-requirements over functional goals with clear-cut quantitative criteria such as `99.999\%' reliable - a probabilistic value to make it compatible with the PMC estimation results.

To perform the , we focus on dependability related metrics that should be estimated and compared to their required constraint values through quantitative analysis. Sensitive analysis to reveal how different system parts contribute to the overall value of those attributes. Sensitive analysis may be considered analogous to the original GORE contribution analysis.



\section{TROPOS to PMC Code Generation}

As we wanted to automate the code generating process for the verification model, the graphical modelling environment that supports TROPOS methodology and the code generation for multi-agents was extended to also generate probabilistic models for the PMC technique.

To reduce the effort of codifying the verification model, an automated generation of the PRISM probabilistic model was implemented based on an existing open source tool for TROPOS development support named TAOM4E[citation]. TAOM4E provides a graphical environment for goal modelling with TROPOS methodology based on the well known Eclipse Modelling Framework (EMF) and Graphical Editing Framework (GEF). The GORE to PRISM generator was implemented as a Eclipse plugin and integrated to the TAOM4E environment. 

The purpose  of the automated code generation for the probabilistic PRISM model is to optimize the formal verification step by abstracting the PRISM language from the analysts and reduce the overhead and time of the model verification. This should increase the feasibility of adopting the extended TROPOS methodology by keeping analysts with their original responsibility of modelling and analysing the system, its social environment and its different contexts of operation.






In terms of a high level system behaviour, each activity has its own states space including success and failure. Our probabilistic verification approach requires not only a formal specification of the system behaviour, but also metrics related to how individual components involved in system activities will perform in respect to the analysed metric. In reliability verification, each component has an individual probability of successfully performing its functional task. Analysts must obtain these values by consulting their manufacturer, by individually analysing each component reliability based on their behaviour specification until the atomic level or by monitoring these components in a testing or production environment. Further details of how individual metrics may be obtained for the PCM may be found at the literature and are out of the scope of this work.
