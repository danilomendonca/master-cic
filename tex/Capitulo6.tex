\chapter{Automatic PRISM generation}\label{ch:implementation}


As we wanted to automate the code generating process for the verification model, the graphical modelling environment that supports TROPOS methodology and the code generation for multi-agents was extended to also generate probabilistic models for the PMC technique.

To reduce the effort of codifying the verification model, an automated generation of the PRISM probabilistic model was implemented based on an existing open source tool for TROPOS development support named TAOM4E[citation]. TAOM4E provides a graphical environment for goal modelling with TROPOS methodology based on the well known Eclipse Modelling Framework (EMF) and Graphical Editing Framework (GEF). The GORE to PRISM generator was implemented as a Eclipse plugin and integrated to the TAOM4E environment. 

The purpose  of the automated code generation for the probabilistic PRISM model is to optimize the formal verification step by abstracting the PRISM language from the analysts and reduce the overhead and time of the model verification. This should increase the feasibility of adopting the extended TROPOS methodology by keeping analysts with their original responsibility of modelling and analysing the system, its social environment and its different contexts of operation.





In terms of a high level system behaviour, each activity has its own states space including success and failure. Our probabilistic verification approach requires not only a formal specification of the system behaviour, but also metrics related to how individual components involved in system activities will perform in respect to the analysed metric. In reliability verification, each component has an individual probability of successfully performing its functional task. Analysts must obtain these values by consulting their manufacturer, by individually analysing each component reliability based on their behaviour specification until the atomic level or by monitoring these components in a testing or production environment. Further details of how individual metrics may be obtained for the PCM may be found at the literature and are out of the scope of this work.


In the PMC technique that has been adapted by this proposal, a behavioural specification, usually provided by UML activity and sequence diagrams, are manually converted to a probabilistic model in PRISM language. This tends to be a costly and error-prone process with a complexity proportional to the number of components, actions and interactions causing state transitions.  

As a goal model is traversed from strategical root goal to operational leaf-goals, and each leaf-goal is reachable by a delegation to other actor or by a operational task, then a behaviour specification as proposed by the RGM may have enough information to be consumed as input for the generation of probabilistic models in PRISM language and for the verification of some important NFRs. However, the manual generation from RGM is still a costly task.

Depending on the abstraction level and the nature of the verification, PRISM models may either very complex or may follow a clear pattern. For instance, PRISM modules can be used to represent leaf-tasks of a runtime goal model. Considering a DTMC model, a task workflow can be modelled as a sequence of probabilistic state transitions according to the behavioural specification parsed from the RGM.

This probabilistic model follows a pattern that motivated the implementation of an automatic generation of DTMC models representing leaf-tasks execution directly from a runtime goal model.