\chapter{Introduction}\label{ch_introduction}%

Among the different causes that lead a system to fail, some can be tracked back to design decisions in early system development process, others are caused by variations in the context of operation. Dynamic contexts increases even further the complexity of the development activities, as different contexts may change what the system should accomplish and the available means to do it. Also, some failures are only activated in specific contexts, posing a new threat to the development of dependable systems.

According to Finkelstein et al.~\cite{Andrea01aframework}, contexts are the reification of the system environment and the environment is whatever over which the system has no control and surrounds its operation - battery status, signals strength, sensors availability, infrastructure restrictions, user characteristics, physical environment conditions, etc. A self-adaptive system should be able to adapt to different context conditions to avoid deviations from its specified behaviour, i.e., to avoid failures.

Dependability defines the system ability of delivering a service that can justifiably be trusted. Different dependability attributes are used to characterize the correct system behaviour, its ability to undergo modifications and the consequence of failures. In dependability analysis, fail forecasting should provide an evaluation of the system behaviour in respect to fault occurrence or activation, while fault removal includes the verification, diagnosis and correction of faults~\cite{Laprie2004}. 

A systematic requirements engineering process aims to improve the quality the delivered systems. However, not many methodologies and frameworks investigate the conformance of the system-to-be to qualitative goals and metrics related to system failures, nor provide adequate means for the monitoring and analysis of these metrics as part of a self-adaptive loop. We argue that, despite the robustness of self-adaptation architectures, a faulty or biased analysis may result in severe or catastrophic system failures.


%based on its design model, relying only on later monitoring and testing.

% both strategically through higher level goals and operationally through lower level goals and tasks

%For some systems, late fault correction may be very costly and, in the worse case, may only happen after the manifestation of a catastrophic failure~[DEPENDABILITY].

\section{Problem Definition}

%Goal models are useful for capturing the intentionality behind system requirements. Goal models provide an abstraction layer for the representation of stakeholders as actors and their needs as goals, tasks and resources dependencies. Also, different kinds of decompositions and relations among intentional elements constitute a goal model diagram~\cite{Yu1996, Dardenne1993, tropos}.

In recent works, goal-oriented requirements engineering (GORE) has been adopted for the development of self-adaptive systems. Among others, goal models are used for deriving a self-adaptive architecture~\cite{Tang:2009} and a high-variability design~\cite{Yu:2008}. Also, goals become live requirements entities that can be self-adapted according to the context~\cite{Baresi:2010:1,Baresi:2010:2} or are complemented with a special class of meta-requirements that refers to their success/failures and other metrics that must be satisfied through self-adaptation~\cite{Souza:2011}. 

Goal models are not restricted to higher-level, strategical goals. Through AND/OR-decomposition, goals are further detailed and means-end tasks are responsible for the operationalization of leaf-goals. Thus, tasks may be directly mapped to activities that composes the system behaviour. Dalpiaz et al. proposed a paradigm shift for the goal modelling and analysis. Instead of a static model for design-time analysis, a regular expression for behaviour specification over goals and tasks composes the runtime goal model (RGM). RGM is suitable for verifying the conformance of the execution and fulfilment of monitored tasks and goals instances to their class specification~\cite{Dalpiaz:2013}.

Despite its contribution to the runtime conformance verification, detailed measurement of the success and failure rates over temporal frames are not yet supported by the RGM framework. Also, RGM relies on measurements over the past execution like `the percentage of success for a given goal over the past month' and `the trend for failure of a given goal in the last week'. Due to this limitation, the original RGM approach is not appropriate for proactive self-adaptation in which systems should avoid violations by detecting or estimating the probability of failures in future executions.

Given the fact that for many systems the ability to avoid failures before they take place is crucial and considering the benefits of a goal-oriented approach for requirements engineering (and for architecture and design, according to~\cite{Bresciani:2004}), in this work we investigate the feasibility of a formal verification for the goal model representation of a system in order to estimate the probability of different goals in being fulfilled as part of a proactive self-adaptation analysis. Accordingly, the first and more general research question emerges:

%Considering a system goal model with additional runtime regex (RGM) and context notations (CGM), this work aims to answer the following questions:

\bigskip

\setlength{\fboxsep}{10pt}
\noindent\fbox{%
    \parbox{\textwidth}{%
        \textbf{Research question 1 (RQ1):}	Given a correct and consistent goal model for which goals are ultimately satisfied by system tasks, is it possible to analyse the probability of different goals in being fulfilled?
    }%
}\bigskip


%Additionally, the RGM proposal do not consider the contexts variations effects over goals, means and qualitative metrics, as described  by the contextual goal model~\cite{Ali:2010}. 


%Thus, it is limited to reactive self-adaptation and do not provide future estimations for these metrics, as a proactive self-adaptation requires. 



%~\cite{Morandini:2008, Cheng:2009} without a specific focus on dependability attributes. Others, such as  and MUSIC~\cite{Hallsteinsen:2012}, propose self-adaptive architectures that lack an explicit representation of system goals. Accordingly,   

%In traditional Goal-oriented requirements engineering (GORE), contribution  analysis is based on domain knowledge about the positive, neutral (implicit) or negative impact of a given system alternative to one or more system goals, generally a qualitative goal. By  comparing the overall contribution of two or more alternatives, a design-time decision is made about which one should be adopted for the system-to-be. 

%For instance, if one goal is to communicate with a remote user mobile, alternative means for the notification agent could be to send a SMS, an internet based message or a voice call. These alternatives may contribute with different values for qualitative goals such as `reliable delivery', `fast delivery', `convenient delivery', etc.  


%Contribution analysis is limited in many ways. First, it is based on domain knowledge information that may not exist or may not be precise and reliable. As a consequence, the decision of which alternative to use may be biased and lead to unexpected violations - the selected alternative may prove to be inappropriate or to fail. From this, our first research question emerges:
%
%%In TROPOS, a contribution analysis aims to select the best alternative based on positive or negative contribution to  softgoals.
%
%\bigskip
%
%\setlength{\fboxsep}{10pt}
%\noindent\fbox{%
%    \parbox{\textwidth}{%
%        \textbf{Research Question 1}		Is it possible to use a more precise verification of non-functional verification in a goal-oriented methodology?
%    }%
%}\bigskip
%
%Second, contribution analysis is based on a static representation without any runtime specification required for estimating non-functional metrics that depend on system behaviour and its many nuances such as temporal order, cardinality and priority. From this, our second research question arises: 
%\bigskip
%
%\noindent\fbox{%
%    \parbox{\textwidth}{%
%        \textbf{Research Question 2}		Is it possible to complement a goal model with a behaviour specification concerning, e.g., the temporal order, cardinality and priority of the achievement of goals and the execution of tasks?
%    }%
%}\bigskip
%
%Third, contribution links are deterministic. As such, the design decisions based on contribution analysis are reduced to a simple sum comparison of the concurring alternatives contributions with no support for probabilistic verification. Accordingly, our third research question is:
%\bigskip
%
%\noindent\fbox{%
%    \parbox{\textwidth}{%
%        \textbf{Research Question 3}		Is it possible to use a probabilistic approach for the verification of non-functional requirements?
%    }%
%}\bigskip
%
%Fourth, TROPOS provides no support for the estimation of non-functional metrics in temporal frames such as ``the percentage of success for a given goal in a month''. This leads us to our fourth research question:
%\bigskip
%
%\noindent\fbox{%
%    \parbox{\textwidth}{%
%        \textbf{Research Question 4}		Is it possible to estimate time-bounded non-functional metrics related to the achievement or failure of system goals?
%    }%
%}\bigskip
%
%Additionally to the contribution analysis limitations, the context in which systems operate may not be static. Mobile and pervasive computing, among others, are examples of new computer paradigms for which the environment is dynamic. Battery, signals strength, components availability and the quality of physical resources and relevant information such as the user geographic location may vary through time, posing a new challenge to the development of socio-technical systems based on these paradigms. The contextualization of the informations gathered at RE phase becomes imperative once its validity may be threatened by changing environment conditions~[CGM]. From this, our fifth and last research question arises:
%\bigskip
%
%\noindent\fbox{%
%    \parbox{\textwidth}{%
%        \textbf{Research Question 5}		Is it possible to consider the context effects described by the CGM in the verification process?
%    }%
%}\bigskip


%for these systems must also address the problem of multiple contexts of operation.

%The contextualization of an information means that its validity is not absolute in respect to the state of the world it relates to.


Model checking is a formal technique to automatically verifying properties over a finite-state model that can represent different system aspects, e.g., its behaviour in the form of components interaction or a higher-level activities workflow. The advantage is to provide, through an appropriate verification model and properties defined with linear, temporal order or even probabilistic logic, complex querying capabilities about the correctness of the system model and its conformance in fulfilling both functional and non-functional requirements.

In specific, the probabilistic model checking (PMC) has been largely explored and is supported by tools such as PRISM model checker. As long as the verification model built from a behaviour specification is precise, this method provides a trusted estimation for metrics as those related to dependability attributes. For instance, PMC can estimate the probability of a system to reach its final success state in a DTMC model based on the reliability of the components involved in the execution, which defines the global reliability of this system or for the analysed activity~\cite{Baier:2008, Nunes:2012}.

Following a similar principle of the component-based verification, the investigation focus has been narrowed to the feasibility of a goal-oriented PMC approach for which the leaf-tasks representing high-level system activities are mapped to a probabilistic model to provide both qualitative and quantitative analysis for the probability of successfully fulfilling different system goals. Accordingly, a second and more specific research question arises:

\bigskip

\setlength{\fboxsep}{10pt}
\noindent\fbox{%
    \parbox{\textwidth}{%
        \textbf{Research Question 2 (RQ2):}	Given a correct and consistent goal model for which goals are ultimately satisfied by system tasks and a set of leaf-tasks that fulfils some specific system goal, is it possible to analyse the probability of achieving this goal through a probabilistic model checking technique?
    }%
}\bigskip

%Given a set of leaf-tasks that fulfils a system goal, is it possible to estimate the probability of achieving this goal through probabilistic model checking

%The context in which systems operate may not be static. Mobile and pervasive computing, among others, are examples of new computer paradigms for which the context of operation is dynamic. Battery, signals strength, components availability and the quality of physical resources and relevant information such as the user geographic location may vary through time, posing a new challenge to system development.

As described by the contextual goal model~\cite{Ali:2010}, the contextualization of the informations gathered at RE and at design phases becomes imperative once its validity may be threatened by changing environment conditions. Context variation may relativize the need for a goal, restrain the adoptability of alternative means and also affect the quality of available means in fulfilling their goals. Hence, a goal-oriented verification must consider these effects in the analysis results. From this, our third research question arises:

\bigskip

\noindent\fbox{%
    \parbox{\textwidth}{%
        \textbf{Research Question 3 (RQ3):}		Is it possible to consider the context effects over what goals are required, what alternatives are adoptable and the quality of each alternative in the probabilistic verification process?
    }%
}\bigskip

As discussed earlier, self-adaptation must rely on an adequate analysis that conforms to the complexity and criticality of the metrics being analysed. Reliability is an important dependability attribute and a first class requirement for self-adaptation, as it defines the continuity of a correct service, i.e., the absence of failures. In goal models, runtime variability solving is based on runtime inputs and non-functional criteria measured or estimated for each alternative~\cite{Yu:2008}. Considering a self-adaptation analysis for solving variability, our third research question is defined as:

\bigskip

\setlength{\fboxsep}{10pt}
\noindent\fbox{%
    \parbox{\textwidth}{%
        \textbf{Research Question 4 (RQ4):}	Is it feasible to employ a goal-oriented reliability verification based on PMC at runtime as part of a self-adaptation analysis for solving variability?
    }%
}\bigskip


\section{Proposed Solution} 

%\bigskip
%
%\setlength{\fboxsep}{10pt}
%\noindent\fbox{%
%    \parbox{\textwidth}{%
%        \textbf{Research Question 1}	Given two or more OR-decomposed alternatives for the same goal, how each alternative affects the probability of successfully achieving this goal in future executions?
%    }%
%}\bigskip


%for which a behavioural model, usually specified in UML activity and sequence diagrams, serve as input for the generation of a probabilistic verification mode

%provides qualitative and quantitative reliability analysis, for the reliability verification of a goal model. In contrast to the previous approach, our purpose would be to represent leaf-tasks as activities of the highest 

To the best of our knowledge, a goal-oriented approach for dependability analysis based on a formal verification method have not yet been proposed. In order to achieve the probability estimation of the fulfilment of different system goals, it is important to realize that the primitive goal models as proposed in~\cite{Dardenne1993, Yu1996, Bresciani:2004} are designs artefacts and, as argued by Dalpiaz et al., not tailored for runtime analysis. Nonetheless, the RGM filled this gap by adding a behavioural specification to the static goal model~\cite{Dalpiaz:2013}.

In contrast to previous approaches for dependability analysis through PMC, this work proposes a verification that is directly mapped to a system runtime goal model. Instead of building the probabilistic verification model from behaviour models like UML activity and sequence diagrams, we benefit from the RGM syntax that specifies the behaviour for goals and tasks to build a high-level DTMC model for reliability verification of different system goals (RQ2). This model, comparable to a high-level UML activity diagram, should also include the context effects over goals, means and qualitative metrics as defined by additional context notations in the goal model (RQ3).

PRISM tool provides a rich analysis environment for PMC. However, a runtime self-adaptation analysis must be automatic, i.e., based on computable processes without human intervention. The parametric PMC answers this requirement by generating a parametric formula for a given probabilistic model and property to be analysed with parameters in the place of constant variables. Thus, different runtime analysis may be performed by just initializing the parameters with values corresponding, e.g., to a specific context of operation, and evaluating the resulting formula. In our proposal, we investigate the feasibility of a parametric PMC as the formal method for the goal-oriented reliability analysis in self-adaptive systems (RQ4).

%for a goal-oriented reliability verification of self-adaptive systems, we aim to extend an existing GORE methodology that was originally restricted to the development of systems with no support for self-adaptation. The idea is to benefit from the goal modelling rationale and environment for requirements engineering, detailed architecture and design of the TROPOS methodology and contribute with an integrated approach for the dependability analysis as part of the self-adaptation loop based on a probabilistic model checking technique.

%PRISM is a model checker tool that supports probabilistic computational tree logic (PCTL) property verification. PCTL provides the syntax for time-bounded probabilistic properties such as ``what is the probability of failure of a given goal in the next x time units?''. This is useful for scenarios with restrictions affected by time, like the battery of a mobile device. The use of PCTL properties addresses our RQ3 (probabilistic) and RQ4 (temporal).


%The advantage of this approach is to reduce the correction costs by anticipating failures at early stages of the development cycle using not the system itself, but its representation. 


%a formal method for static verification of different properties, including non-functional metrics like reliability. In order to verify non-functional requirements for the system-to-be, a corresponding probabilistic model of the system behaviour must be built. In this work, the RGM proposed by Dalpiaz et al. is used as input for the generation of a DTMC model and non-functional requirements are verified with the PMC technique proposed by Uriel~[URIEL]. This probabilistic approach addresses our RQ3.


%Goal models are not restricted to strategical goals. Through AND/OR-decomposition, goals are further detailed and means-end tasks are responsible for the operationalization of leaf-goals. Thus, tasks may be directly mapped to activities that composes the system behaviour. A probabilistic, model-based verification of a goal model, as it will be explained in later sections, addresses our RQ1.

%Increased analysis overhead is one of the drawbacks of this approach. Thus, its application must be well justified by the criticality level of the system and by the adoption of an appropriated verification approach. 

%As the goal model proposed in TROPOS is static, no information regarding achievement/execution order, cardinality and priority of goals/tasks is available, except the activity diagram for the detailing of an agent's single capability behaviour and the sequence diagram for agents interaction. Nonetheless, this problem was tackled by Dalpiaz et al. with a regular expression language to specify the behaviour in a goal model, e.g., how many times the same goal should be achieved and the execution order of different system tasks~[RGM]. This answers our RQ2. 

%Despite its valuable contribution to the requirements analysis and engineering, GORE still lacks the proper means to verify the conformance of some important metrics such as dependability requirements and others non-functional requirements (NFR) related to the likelihood and frequency of system execution failures. By tackling the verification of these metrics with a more precise and reliable approach, this work aims to mitigate the occurrence of failures at an early phase of the system development that must justifiably be trusted.

%In order to provide a more solid and precise approach for the non-functional verification of different system alternatives and to improve the GORE contribution analysis, we propose the extension of the TROPOS goal-oriented software development methodology with 



%IN PROPOSAL
%We have used the RGM proposal to fill the gap between the static goal model and its dynamic representation. This dynamic view of the goal model is translated to a probabilistic model following the PMC technique and the PRISM model checker as the tool to automate the verification. PRISM language is used to define the probabilistic model and the Probabilistic Computation Tree Logic (PCTL) to describe the properties to be verified in the model. These properties are derived from the NFR associated to corresponding system goals being verified.

%Finally, to address the problem of a dynamic context of operation, the context effects over goals, means and metrics should be treated by the verification model. Context variables may be parametrized to produce a formula that can check the system and its alternatives for different contexts. The feasibility of this approach, as presented by later chapters of this work, answers RQ5. 

%By coupling a formal verification to a goal model, our approach benefit from a clearer understanding of the system-to-be and its criticality, justifying the verification of system parts or the system whole. The automatic generation of the verification model from an extended version of the goal mode, namely the runtime goal model, greatly reduces the verification overhead. Finally, including context effects provide a more realistic representation of the system to be verified.

%Runtime self-adaptation is beyond the scope of this work. However, based on the contextual analysis provided by the CGM and the enriched non-functional and dependability analysis provided by the verification of different alternatives using the PMC technique, it should not be difficult to extend the approach with the additional monitoring, planing and execution capabilities of a self-adaptation loop and have a self-adaptive architecture and mechanism reflected upon its runtime goal model requirements. These concerns should be addressed in future work.

%This space variability problem becomes more complex with the contextualization of goals, means and metrics. According to Bosh et al.[cite Bosh 2004], a high degree of variability allow the use of software in a broader range of contexts. Traditional GORE approaches are mostly used to the selection of which single alternative would exist in the system-to-be for some static context. A dynamic environment can result in contextual violations that can not be resolved just by configuration management. In these cases, a space variability may be required to keep the system properly running in different contexts.


%to a specific kind of goal called \textit{softgoal}. Softgoals are goals for which there is no clear-cut criteria. Often, they represent qualitative intentions of stakeholders, in contrast to the 

\section{Evaluation}

The proposal has been evaluated with the application of the extended TROPOS methodology for context-aware, self-adaptive systems to the development of a Mobile Personal Emergency Response System (MPERS). This system may be seen as a body area network (BAN) with extended functionalities related to ubiquitous emergency response running in a mobile device [BAD]. Instead of static environment, the MPERS is conceived to allow patients with different health risk degrees to preserve their mobility while they are monitored and assisted. For being a mobile system, MPERS is affected by context variations and self-adaptation becomes a mandatory feature to avoid failures, as safety is of paramount importance for this system.

The MPERS features were based on real emergency response systems available at the industry and also at the BAN explored in previews works~\cite{Lorincz2004, Nunes:2012, MPERS:GoSafe}. In this evaluation, we focus on the development process of the MPERS and demonstrate the feasibility of the goal-oriented approach for the dependability analysis with a probabilistic model checking technique. In specific, our main goal is to provide reliability analysis capability for the MPERS system with a verification model mapped to the contextual/runtime goal model of the system-to-be, empowering both the analysts (design-time) and the system itself (runtime self-adaptation) with a precise method for reliability verification coped to the goal model specification.

%If a medical emergency is detected, a geolocation feature should point out the location where the emergency response team must be addressed to. 

%This evaluation process should also reveal the major benefits and limitations of the proposal. Time to market is an important aspect for any software development methodology. Also, the soundness and precision of the proposed probabilistic verification is crucial and must be evaluated as they should not result in mislead decisions about which alternatives should be used by the system. Instead, they must anticipate any violation that could lead to a system failure, specially severe or catastrophic failures, giving analysts valuable information about where the system requirements and specification should be tailored and improved.

\section{Contributions Summary}

This section summarizes the contributions of this proposal.

%A seguir um resumo das con

\begin{enumerate}

\item An goal-oriented approach for reliability analysis based on probabilistic model checking.

\begin{itemize}

\item Use of a probabilistic model checking whose verification model is built from a runtime goal model with leaf-tasks as system activities.
\medskip

\item Inclusion of the context effects over goals and tasks in the verification model as defined by the contextual notations in the runtime goal model.
\medskip

\item Probabilistic and temporal querying capabilities for different metrics related to the reliability of one or more goals in the system goal model.
\end{itemize}
\bigskip

\item An automated generation of the DTMC model in PRISM language from a contextual/runtime goal model.
\begin{itemize}

\item A parser implementation for the regular expression (regex) language used in runtime goal models with support for execution order, cardinality, alternative execution, optional execution, conditional execution and multiple parallel or sequential executions of the same task. 
\medskip

\item Definition of conversion rules between different decomposition types and behaviours rules in the contextual/runtime goal model to a DTMC model in PRISM language.
\medskip

\item JAVA implementation of DTMC generator integrated to the TAOM4E tool that supports the TROPOS methodology using the Eclipse plugin architecture.
\end{itemize}

\end{enumerate}

\section{Document Organization}

This dissertation is organized as follows. Chapter~\ref{ch:baseline} presents the base concepts of this work. Chapter~\ref{ch:related_work} describes the most important related works. Chapter~\ref{ch:motivation} details the motivation for this work and the requirements for the proposed goal-oriented reliability analysis. Chapter~\ref{ch:proposal} describes and evaluates the  proposal with the MPERS cases study. Chapter~\ref{ch:implementation} presents the rules for the automatic generation of a DTMC model in PRISM language from the runtime goal model with additional context effects notations. Chapter~\ref{ch:validation} validates the proposal with some relevant metrics. Finally, Chapter~\ref{ch:conclusion} concludes this work with final considerations about the current proposal and our future work.