\chapter{Baseline}\label{ch_baseline}%

\section{Goal-oriented Requirements Engineering}

Goal-oriented requirements engineering brings forward the intentionality behind system requirements. More than just presenting the \textit{what} and the \textit{how} of a system-to-be, it provides the justification for each requirement, that is, they also present the \textit{why}. Through a directed graph tree that begins with a root goal, goals are connected trough decomposition links. Root and higher level goals are related to strategical concerns, while lower level and leaf-goals are related to technical and operational features of the system. 

The main purpose of a goal model is to support the early process of RE, including the elicitation of social needs and dependencies, the actors involved in delivering functionalities and resources, the decomposition of higher-level goals into more granular and detailed requirements chunks, the operationalization through means-end tasks and finally the comparison between different alternatives for the system-to-be. A goal model is said to be valid and complete if it follows all its syntactic rules and if all system goals are either decomposed, delegated to other actors or fulfilled by operational system tasks. 

Three frameworks/methodologies, namely KAOS, i* and TROPOS, represent the foundations for the goal model analysis used by a variety of other proposals [KAOS, i*, TROPOS]. Despite some differences among their syntax, they all share a set of core concepts:
\medskip

\large{\underline{Entities}}

\begin{itemize}

\item \textbf{Actor:} an entity that has goals and can decide autonomously how to achieve them. They represent a physical, social or software agent. E.g.: A patient, an emergency center, a doctor and a Mobile Personal Emergency System running in patient's smartphone.
\medskip

\item \textbf{Goal:} are actors' strategic interests. A goal with a clear-cut criteria for its satisfaction is called a hard goal. In opposition, softgoals has no clear-cut criteria for deciding whether they are satisfied or not and are usually associated to non-functional requirements of an actor. E.g.: vital signs are monitored, emergency is detected, emergency center is notified (hard goals) and system availability, detection precision, emergency awareness (softgoals).
\medskip

\item \textbf{Task:} an operational means to satisfy actors' goals. E.g.: monitor temperature sensor, persist vital signs data, request emergency assistance.

\end{itemize}
\medskip

\large{\underline{Relations}}

\begin{itemize}

\item \textbf{AND/OR Decomposition:} a link that decomposes a goal/task into sub-goals/sub-tasks, meaning that all (at least one) decomposed goal(s)/task(s) must be fulfilled/executed in order to satisfy its parent entity. 
\medskip

\item \textbf{Means-end:} a means to fulfil an actor's goal through the execution of an operational task by the same actor.
\medskip

\item \textbf{Contribution link:} a positive or negative contribution between a given goal/task to a softgoal. Contribution links are used for deciding between alternative goals/tasks at design time (contribution analysis).

\end{itemize}

\section{TROPOS Goal-oriented Software Development Methodology}

TROPOS is a GORE methodology based on the i* framework. Its main improvement is the addition of new phases of requirements engineering and system design, namely~[TROPOS]:

\begin{itemize}

\item Late requirements engineering: Beyond the social dependency modelling with actors diagrams representing stakeholders and their needs in early requirements phase, a late requirements phase focuses on the system actor analysis. In this phase, system goals are inherited from stakeholders needs and represent both functional and non-functional requirements. Each goal has to be further decomposed in more granular sub-goals, delegated to other actors or to be fulfilled by means-end tasks. 
\medskip

\item Architectural design: In this phase, new actors representing sub-systems are created to fulfil different system goals. The idea is to shape the solution using a multi-agent architecture style instead of a monolithic system approach. Data and control interconnections are represented as dependencies.
\medskip

\item Detailed design: The last phase is characterized by the specification of agent capabilities and interactions though UML activity and sequence diagrams. Also, the implementation platform and other specific implementation details are addressed in order to directly map the design to system code.

\end{itemize}

Implementation phase is also specified by TROPOS methodology, but it is out of the scope of this work as our objective is to improve the analysis and the solution that will be later implemented.

\section{Contexts}

Context may be defined as the reification of the environment that surrounds the system operation~[FINKElSTEIN]. Contexts, as already stated, may not be static, but dynamic. A system has no control over its context of operation. Accordingly, a system must be able to support different contexts of operation without violating its goals. Moreover, systems should be able to monitor the state of its surrounding environment and decide which alternative will be used regarding both the availability of that alternative and the optimization of non-functional requirements.

In GORE, dynamic contexts may affect what goals a system have to reach, the means available to meet them and also the quality achieved by each alternative[CGM]. Root goal and higher level strategical goals are not contextualized as they represent the main purpose of a system [Finkelstein]. As these goals are decomposed in more granular sub-goals, a context condition may dictate:

\begin{enumerate}

\item If the goal is required for that context, limiting `what' a system should do;
\medskip

\item If a sub-goal or task is adoptable, limiting the `means' to fulfil a required goal;
\medskip

\item The positive, neutral or negative contribution of using some goal or task to another goal, usually a qualitative softgoal;

\end{enumerate}

The third effect is the main focus of this work, as it is related to the GORE contribution analysis that we aim to improve.
 

\section{Variability Solving}

In goal models, OR-decomposition may lead to more than one path that, if its tasks are executed, goals satisfied and dependencies met, will satisfy the root goal of a system. This variability problem has different natures and must be treated accordingly:

\begin{itemize}

\item Different paths, one context, one solution: Despite the elicitation of more than one path, only one alternative will be kept. Softgoals and non-functional metrics are generally used as criteria for selection.
\medskip

\item Different paths, multiple contexts, multiple  solutions: A system affected by context variation may have to select a different alternative for different contexts, justifying a system-to-be with multiple solutions. 

\end{itemize}

In the first case, NFR verification should point out at design time which alternative paths also conforms to the non-functional metrics associated to the goal model in addition to the softgoal contribution analysis. If more than one metric exists, some multi-criteria approach should decide which alternative will be selected. 

In the last case, NFR verification have two purposes: to select the best single alternative for a given context at runtime, or to measure the joint set of alternatives for any non-functional metric at design time. The former may follow a similar multi-criteria approach if more than one metric exists. The later corresponds to the family verification of a software product line~[GENAINA].

\section{Dependability Analysis}

The concept of dependability is related to dependence and trust as well as the ability of a system to avoid failures that are more frequent and more severe than certain threshold~[AVIZIENIS]. According to Avizienis et al., dependability encompasses the following attributes: 

\begin{itemize}

\item Availability: readiness for correct service.
\medskip

\item Reliability: continuity of correct service.
\medskip

\item Integrity: absence of improper system alterations.
\medskip

\item Safety: absence of catastrophic consequences on the user(s) and the environment.
\medskip

\item Maintainability: ability to undergo modifications and repairs.
\medskip

\end{itemize}

%Correctness is opposed to failures. 

A holistic dependability specification has to include not only the software operation, but also the requirements for which that operation is meant. Requirements are an important factor to decide the acceptable frequency and severity of a software failure. Similarly, context is another factor in that decision. The frequency and the likelihood of failures are related to the dependability attributes of reliability, availability and integrity. The severity of failures consequence is related to safety.

An execution failure is a perceived deviation from system expected behaviour that may have variable degree of consequence on the user(s) and the environment. These failures are caused by specification faults or specification violations. In the first case, requirements model fails to describe the system and either the goals or the means to fulfil then are incorrect or incomplete. In the second case, software or hardware behaviour did not follow its specification due to a natural phenomena, a human-made, a malicious or an interaction fault~[AVIZIENIS].

%Avizienis et al. also characterizes a failure according to four viewpoints. For this work, failure domain and failure consequence are important as they define .

The scope of this work is restricted to specification violations, i.e., we assume that a system specification is complete and consistent. Failures are restricted to anomalous behaviour of the components participating in the execution of system tasks, including technical components and human actors. Regarding the different means to attain dependability, our proposal consists of a fault forecasting as part of the Validation \& Verification phase of RE by estimating metrics related to dependability and assuring their conformance to the non-functional constraints associated to the goal model. 

%
%At early project phases, hazard resolution may involve simply getting more information about hazards or generating alternative design solutions~\cite{Leveson:1995}. In our work, we have used a qualitative means to analyse the dependability to be delivered by goals of a certain system taking into account contextual effects. Our approach improves the understanding of systems fault-causality effect and the identification of best approaches to reduce risk or even determine rates for safety or system level functional failure. Moreover, dependability requirement analysis cannot be accurately fulfilled without taking into account the context under which the system will operate.
%
%
%Avizienis et al \cite{Laprie2004} proposed a failure classification taxonomy with four viewpoints characterizing failures. In our approach, we use two categories: domain and consequence. We use the domain category to distinguish \textit{content} failures from \textit{timing} failures:
%
%\begin{itemize}
%
%\item{\textbf{Content failure}: When the content of the information delivered by a system task deviates from its specification}
%
%\item{\textbf{Timing failure}: When the time of arrival or the duration of the information delivered by some system task deviates from its specification}
%
%\end{itemize}
%
%The consequence of failures enables the definition of failures' severity. Two limiting levels are predefined and other intermediary levels could be defined for each case:
%
%\begin{itemize}
%
%\item{\textbf{Minor failure}: The harmful consequences of failures are limited or at most similar to the benefits provided by the correct operation of the system}
%
%\item{\textbf{Catastrophic failure}: The harmful consequences of failures are incommensurably higher than the benefits provided by correct operation of the system}
%
%\end{itemize}
%
%More details on the complete failure classification can be found in \cite{Laprie2004}. In our approach, a part of this taxonomy is used to guide the definition of the classes of failures severities and therefore the required level of dependability for different system goals. It also takes part in the identification of which dependability attribute is related to each contextual failure occurrence.  

\section{PRISM Probabilistic Model Checker}

A model checking is a formal method that aims to automatically verify if a system model meets its specification for defined properties. Probabilistic and state based model checking supports the verification of finite-state probabilistic models such as discrete-time Markov chain (DTMC), continuous-time Markov chain (CTMC) and Markov decision process (MDP), among others. Different types of properties may be defined to verify a system model for different qualitative metrics.

The PMC technique used in this approach is supported by the PRISM model checker tool~[PRISM]. PRISM allows the modelling and analysis of systems which exhibit random or probabilistic behaviour. The decision of using PRISM as the probabilistic state based model checker was due to the number of successful case studies that have used this tool, indicating its maturity [PRISM CS], and also due to its rich environment that is able to represent different kinds of probabilistic models and their evaluations. 

PRISM is suitable for many different kinds of model evaluations depending on the abstraction level, the type of probabilistic model and the PCTL properties to be analysed. PRISM language offers a rich set of constructs that may represent system modules and components, among others architectural and design configurations. Both qualitative and sensitive analysis are available.

As it will be explained in later sections, goal models may be easily extended with the proper information required for the verification of some important dependability attributes. The objective is to anticipate non-functional dependability violations and to support the decision of which alternatives to use in the system-to-be. A model checking technique should be used for dependability analysis as long as: 

\begin{itemize}

\item A formal system model may be built;
\medskip

\item Properties representing dependability attributes may be defined;
\medskip

\item The analysis overhead is justified, e.g., by its criticality.
\bigskip

\end{itemize}

Finally, PRISM also supports a parametric model verification. That is, instead of providing the final evaluation for a given property, models may use parameters instead of initialized variables and the verification will output a parametric formula whose evaluation will check the model for any valid combination of parameters values.

\section{Mobile Personal Emergency Response System}

The MPERS case study will be further detailed in later sections as the goal models generated by TROPOS methodology are themselves useful for communication purposes. As such, this section will cover some non-functional aspects of this system that justify the use of our formal verification approach.

An emergency response system is a mission-critical system for which failures in achieving its main goals by the time they are required may lead to catastrophic consequences on users, i.e., on patients monitored by the system expecting to be promptly assisted in case of a medical emergency. Accordingly, any stakeholder that wishes to offer a service based on this system will have both ethical and contractual obligations regarding the safety of its product, that is, it must use appropriate means to prevent system failures.

Other dependability attributes such as reliability and availability are metrics over the correctness of system behaviour. MPERS is expected to have a high availability - as it must be ready to respond to an emergency that may happen at any time - and a high reliability - as an incorrect emergency response may lead to death or to costly false-positives. Integrity is a less critical attribute in this case, but must also be addressed as patient privacy may not be violated by disclosing his personal health or geolocation info to unauthorized persons. Maintainability is addressed by the use of a software development methodology and by the ability to update emergency rules remotely at runtime.


Reliability verification of an Ambient Assisted Living System also based on body-area networks though PCM technique was explored by Fernandes~[Fernandes, 2012]. Reliability estimation demonstrates the non-determinism in the verification model that will result in an non-deterministic evaluation result. Moreover, PRISM cost/reward structure is illustrated by the power consumption verification. Analogue approaches could be used for other attributes with non-determinism and cost/reward structures in the model. The value obtained by this quantitative analysis of both attributes must comply with the corresponding non-functional constraint associated to the goal model.

%It will be up to the analyst and stakeholders to define which type of probabilistic model and which PCTL properties must be analysed for each different system. Dependability attributes may be relevant for any sort of system, but are certainly important for systems with some criticality degree, i.e., for those whose failure could have severe or catastrophic consequences for the user(s) and for the environment.

\section{Antlr Language Recognition Tool}

ANTLR or Another Tool for Language Recognition is a open source parser generator for reading, processing, executing or translating structured text or binary files. The main purpose is to automatically generate a parser for a custom language defined in a specific grammar language supported by the tool. The parser can then be imported in any version compatible JAVA project to build and walk parsed trees. 

As a result, any domain-specific language may be specified and then parsed using JAVA methods that will manipulate primitive attributes and objects according to what each parser rule and lexical term means for that language. In our proposal, ANTLR was successfully used to generate the parser for the regular expression language that specifies the behaviour of a runtime goal model (RGM). Further details of the grammar with both parser rules and lexical terms is given in later section.
