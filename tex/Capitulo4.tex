\chapter{Modelling the Problem}

\section{Motivation}

%According to Lamsweerde, a poor requirements engineering (RE) is the major source of system failures. Lack of user involvement, requirements incompleteness, changing requirements, unrealistic expectations and unclear objectives are common causes~[AXEL].

Goal-oriented requirements engineering (GORE) has gained the attention of both academic and industrial practitioners due to its ability to systematically model the intentionality behind system requirements. More than just presenting the `what' and the `how', goal models also express the `why' of different requirements to exist. Its simple graphical notation allows non-technical stakeholders to take part in the analysis process and have a clear view of the system-to-be. Finally, automated model verification should avoid violations of the requirements specification.

TROPOS is a GORE methodology that also includes architectural and  detailed design phases for the development of socio-technical systems. Socio-technical systems provide and control a wide range of daily used services. Often, these systems are responsible for important and even critical requirements whose failures would cause undesirable or intolerable consequences. This requires developers to take dependability into consideration as a first class requirement.

In TROPOS, as in other GORE frameworks, there is no coupling to any specific verification approach for dependability attributes and other non-functional requirements. Contribution analysis is used for the comparison and selection of alternative design solutions based on how each alternative contribute to one or more system goals, usually qualitative softgoals. This approach, however, is not tailored for metrics depending on system behaviour, i.e., on the dynamic coordination of activities and components interactions. 

Probabilistic fault-forecasting aims to derive probabilistic estimates about measures related to the behaviour of a system in the presence of faults. Dependability benchmark enables the characterization of the dependability and security of a system and the comparison of alternative solutions according to one or several attributes~[AVIZIENIS~37]. This benchmark may be achieved by a probabilistic model checking technique coupled to the later phases of the TROPOS methodology, enabling the dependability benchmark of a goal model.

An important success factor for any software development methodology is a reduced overhead to the development effort, including domain knowledge and tools to support its process. PMC may significantly reduce the occurrence of system failures, but it also requires additional knowledge about its modelling language and verification steps. A goal model extended by behaviour specification overlaps with the UML activity diagram used by PMC.  Moreover, an automatic model generation for the PMC is desirable to reduce the know-how and effort for the verification of metrics for the system, justifying the implementation of this generator as an extension of a TROPOS modelling tool.

\section{Requirements}

Based on the identified gap of a formal verification of non-functional requirements in a goal-oriented requirements engineering approach, we defined the following requirements that must be addressed by our proposal:

\begin{enumerate}[R.1]

\item \textbf{Backward compatibility:} Extended TROPOS runtime regex and contextual notation added to the goal model must not modify the existing syntax and semantic of the original TROPOS methodology.

\item \textbf{Optionality:} The use of the probabilistic model checking as a formal verification approach of runtime goal models as an extended TROPOS methodology should be optional and not mandatory.
\medskip

\item \textbf{Model generation:} The probabilistic model representing the activities of runtime goal models should be automatically generated.
\medskip

\item \textbf{Tool integration:} The same development environment tool used for TROPOS modelling and analysis activities should be extended with the runtime regex, context notations and verification model generation.
\medskip

\item \textbf{Static syntax support:} The verification model should be coherent to the static goal model syntax for the AND/OR decomposition of goals/tasks and for the goal-task means-end relation.
\medskip

\item \textbf{Dynamic syntax support:} The verification model should be coherent to the runtime goal model regex including tasks sequential or interleaved execution order, cardinality as well as alternative, optional and conditional tasks execution.
\medskip

\item \textbf{Contextual syntax support:} The verification model should be coherent to the context effects over the activation of goals, the adoptability of sub-goals/tasks and over the individual quality metric of components. 

\end{enumerate}