\chapter{Modelling the Problem}\label{ch:motivation}

\section{Motivation}

%According to Lamsweerde, a poor requirements engineering (RE) is the major source of system failures. Lack of user involvement, requirements incompleteness, changing requirements, unrealistic expectations and unclear objectives are common causes~[AXEL].

Goal-oriented requirements engineering (GORE) has gained the attention of both academic and industrial practitioners due to its ability to systematically model the intentionality behind system requirements. More than just presenting the `what' and the `how', goal models also express the `why' of different requirements to exist. Its simple graphical notation allows non-technical stakeholders to take part in the analysis process and have a clear view of the system-to-be. Finally, automated model verification should avoid violations of the requirements specification.

TROPOS is a GORE methodology that also includes architectural and  detailed design phases for the development of socio-technical, multi-agent systems. Socio-technical systems provide and control a wide range of daily used services. Often, these systems are responsible for important and even critical requirements whose failures would cause undesirable or intolerable consequences. This requires developers to take dependability into consideration as a first class requirement.

In TROPOS, as in other GORE frameworks, there is no coupling to any specific verification approach for dependability attributes and other non-functional requirements. Contribution analysis is used for the comparison and selection of alternative design solutions based on how each alternative contribute to one or more system goals, usually qualitative softgoals. This approach, however, is not tailored for measures depending on system behaviour or more complex analysis techniques.

In a previous work and following, we have extended the contextual goal model to tackled the context variation effects on dependability attributes based on declarative fuzzy logic rules~\cite{Mendonca:2014}. From our experience, we have considered that a more scalable and precise approach for the verification of dependability metrics in dynamic contexts was needed. Probabilistic model checking emerged as a potential option to cope a goal-oriented methodology with a formal method for dependability analysis.

To the best of our knowledge, formal methods have not yet being used for a goal-oriented dependability analysis. As a fault forecasting approach, the purpose would be to characterize the dependability and the security of a system and to compare concurring alternative solutions - in this case, goal model alternatives - according to one or more attributes. In specific, the reliability attribute should be the focus of this proposal and the resulting verification model should estimate and verify reliability related metrics as a design-time, manual analysis and specially as part of a runtime automated analysis as part of a self-adaptation loop. Finally, the context effects over goals, means and metrics as described by the CGM should be considered in the analysis.

%In dependability analysis, fault-forecasting aims to derive estimates about measures related to the behaviour of a system in the presence of faults. Dependability benchmark enables the characterization of the dependability and security of a system and the comparison of alternative solutions according to one or more attributes~[AVIZIENIS~37]. This benchmark may be achieved by a probabilistic model checking technique coupled to the TROPOS methodology, enabling the (probabilistic) dependability benchmark of a goal model system representation with additional behaviour specification. 


%MAYBE INTRODUCTION
%An important success factor for any software development methodology is a reduced overhead to the development effort, including domain knowledge and tools to support its process. PMC may significantly reduce the occurrence of system failures, but it also requires additional knowledge about its modelling language and verification steps. A goal model extended by behaviour specification overlaps with the UML activity diagram used by PMC.  Moreover, an automatic model generation for the PMC is desirable to reduce the know-how and effort for the verification of metrics for the system, justifying the implementation of this generator as an extension of a TROPOS modelling tool.

\section{Requirements for the Goal-oriented and Contextual Dependability Analysis}

Based on the identified gap of a probabilistic verification of non-functional requirements in GORE methodologies, we defined the following requirements that must be addressed by our proposal:

\begin{enumerate}[R.1]

\item \textbf{Backward compatibility:} Extended TROPOS runtime regex and contextual notation added to the goal model must not conflict to the existing syntax and semantic of the original TROPOS methodology.

%\item \textbf{Optionality:} The use of the probabilistic model checking as a formal verification approach of runtime goal models as an extended TROPOS phase should be optional and not mandatory.
%\medskip

\item \textbf{Verification scope:} The verification scope may be restricted to a part of the system (local goals) or encompass the whole system (root goal), according to analysts decision.
\medskip

\item \textbf{Model generation:} The probabilistic model representing the activities of runtime goal models should be automatically generated from a runtime goal model created in TROPOS late requirements engineering phase.
\medskip

\item \textbf{Tool integration:} The TAOM4E tool/plugin used for TROPOS modelling and analysis activities should be extended with the runtime regex, context notations and verification model generation.
\medskip

\item \textbf{Static syntax support:} The verification model should be coherent to the AND/OR decomposition of goals/tasks and to the goal-task means-end relation of the original/static goal model syntax.
\medskip

\item \textbf{Dynamic syntax support:} The verification model should be coherent to the goal/tasks achievement/execution order, to cardinality and to alternative, optional and conditional achievement/execution syntaxes in a runtime goal model.
\medskip

\item \textbf{Contextual syntax support:} The verification model should be coherent to the context effects over the activation of goals, the adoptability of sub-goals/tasks and over the individual quality metric of leaf-tasks selected for analysis. 

\end{enumerate}