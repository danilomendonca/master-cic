%%%%%%%%%%%%%%%%%%%%%%%%%%%%%%%%%%%%%%%%
% Classe do documento
%%%%%%%%%%%%%%%%%%%%%%%%%%%%%%%%%%%%%%%%

% Nós usamos a classe "unb-cic".  Deixe apenas uma das linhas
% abaixo não-comentada, dependendo se você for do bacharelado ou
% da licenciatura.

% Para tirar os comentários, é só mudar o comando para fazer nada.
\newcommand{\com}[1]{\textcolor{red}{#1}}%

\documentclass[mestrado]{unb-cic}



%%%%%%%%%%%%%%%%%%%%%%%%%%%%%%%%%%%%%%%%
% Pacotes importados
%%%%%%%%%%%%%%%%%%%%%%%%%%%%%%%%%%%%%%%%

\usepackage[brazil,american]{babel}
\usepackage[T1]{fontenc}
\usepackage{indentfirst}
\usepackage{natbib}
\usepackage{xcolor,graphicx,url}
\usepackage[utf8]{inputenc}
\usepackage{amsmath,amssymb,amsthm}
\usepackage{footnote}
%\usepackage{minipage}
\usepackage{tablefootnote} 
\usepackage{listings}
\usepackage{enumerate}


%%%%%%%%%%%%%%%%%%%%%%%%%%%%%%%%%%%%%%%%
% Cores dos links
%%%%%%%%%%%%%%%%%%%%%%%%%%%%%%%%%%%%%%%%

% Veja o arquivos cores.tex se quiser ver que outras cores estão
% pré-definidas.  Utilizando o comando \hypersetup abaixo nós
% evitamos aquelas caixas vermelhas feias em volta dos links.

%%%%%%%%%%%%%%%%%%%%%%%%%%%%%%%%%%%%%%%%
% Cores do estilo Tango
%%%%%%%%%%%%%%%%%%%%%%%%%%%%%%%%%%%%%%%%

\definecolor{LightButter}{rgb}{0.98,0.91,0.31}
\definecolor{LightOrange}{rgb}{0.98,0.68,0.24}
\definecolor{LightChocolate}{rgb}{0.91,0.72,0.43}
\definecolor{LightChameleon}{rgb}{0.54,0.88,0.20}
\definecolor{LightSkyBlue}{rgb}{0.45,0.62,0.81}
\definecolor{LightPlum}{rgb}{0.68,0.50,0.66}
\definecolor{LightScarletRed}{rgb}{0.93,0.16,0.16}
\definecolor{Butter}{rgb}{0.93,0.86,0.25}
\definecolor{Orange}{rgb}{0.96,0.47,0.00}
\definecolor{Chocolate}{rgb}{0.75,0.49,0.07}
\definecolor{Chameleon}{rgb}{0.45,0.82,0.09}
\definecolor{SkyBlue}{rgb}{0.20,0.39,0.64}
\definecolor{Plum}{rgb}{0.46,0.31,0.48}
\definecolor{ScarletRed}{rgb}{0.80,0.00,0.00}
\definecolor{DarkButter}{rgb}{0.77,0.62,0.00}
\definecolor{DarkOrange}{rgb}{0.80,0.36,0.00}
\definecolor{DarkChocolate}{rgb}{0.56,0.35,0.01}
\definecolor{DarkChameleon}{rgb}{0.30,0.60,0.02}
\definecolor{DarkSkyBlue}{rgb}{0.12,0.29,0.53}
\definecolor{DarkPlum}{rgb}{0.36,0.21,0.40}
\definecolor{DarkScarletRed}{rgb}{0.64,0.00,0.00}
\definecolor{Aluminium1}{rgb}{0.93,0.93,0.92}
\definecolor{Aluminium2}{rgb}{0.82,0.84,0.81}
\definecolor{Aluminium3}{rgb}{0.73,0.74,0.71}
\definecolor{Aluminium4}{rgb}{0.53,0.54,0.52}
\definecolor{Aluminium5}{rgb}{0.33,0.34,0.32}
\definecolor{Aluminium6}{rgb}{0.18,0.20,0.21}

\hypersetup{
  colorlinks=true,
  linkcolor=DarkScarletRed,
  citecolor=DarkScarletRed,
  filecolor=DarkScarletRed,
  urlcolor= DarkScarletRed
}



%%%%%%%%%%%%%%%%%%%%%%%%%%%%%%%%%%%%%%%%
% Informações sobre a monografia
%%%%%%%%%%%%%%%%%%%%%%%%%%%%%%%%%%%%%%%%
\title{An Extended Goal-oriented Development Methodology with Contextual Dependability Analysis}%

\orientador[a]{\prof \dr[a] Genaína Nunes Rodrigues}{CIC/UnB}
%\coorientador[a]{\prof[a] \dr[a] Coorientadora}{MAT/UnB}
\coordenador[a]{\prof[a] \dr[a] Alba Cristina Magalhaes Alves de Melo}{CIC/UnB}
\diamesano{30}{janeiro}{2015}%

\membrobanca{\prof[a] \dr[a] Vander Alves}{CIC/UnB}
\membrobanca{\prof \dr Luciano Baresi}{Politecnico di Milano}

\autor{Danilo F.}{Mendonça}
\CDU{004.4}

\palavraschave{\LaTeX, metodologia científica}
\keywords{\LaTeX, scientific method}



\graphicspath{{.}{img/}}%
\newcommand{\unbcic}{\texttt{UnB-CIC}}%
%%%%%%%%%%%%%%%%%%%%%%%%%%%%%%%%%%%%%%%%
% Texto
%%%%%%%%%%%%%%%%%%%%%%%%%%%%%%%%%%%%%%%%

\begin{document}
  \maketitle

  \begin{dedicatoria}

  \end{dedicatoria}

  \begin{agradecimentos}

  \end{agradecimentos}

\hyphenation{au-xi-li-ar}

  \begin{resumo}
  A static and stable operation environment is not a reality for many systems nowadays. Context variations impose many threats to systems safety, including the activation of context specific failures. Goal-oriented software-development methodologies (SDM) adds the `why' to system requirements, i.e., the intentionality behind system goals and the means to meet then. Contexts may affect what requirements are needed, which alternatives are available and the quality of these alternatives, including dependability attributes. In order to allow a formal and probabilistic analysis of systems affected by context variation and elicited with Goal-Oriented Requirements Engineering (GORE) approach, we have proposed an extension to the TROPOS goal-oriented methodology to include dependability constraints to goals and to provide a more precise and formal requirements verification by translating a contextual goal-model annotated with a behavioural regular expression into a PRISM probabilistic model to be checked against properties defined with the Probabilistic Computation Tree Logic (PCTL). We evaluated the proposal with a case study of a Mobile Personal Emergency Response System (MPERS).
  
%In this work, we propose the extension of the TROPOS methodology with a probabilistic model checking approach tailored for the verification of dependability requirements regarding multiple contexts of operation as part of the validation & verification phase of RE. 
  
  \end{resumo}

  a\selectlanguage{american}
  \begin{abstract}
  	
  \end{abstract}
  \selectlanguage{brazil}
\hyphenation{a-tri-bu-tos}
  \tableofcontents
  \listoffigures
  \listoftables

\renewcommand{\appendixname}{Anexo}


  \textual
  
  \chapter{Introduction}\label{ch_introduction}%

\section{Problem Definition}

Among the different causes that lead a system to fail, some can be tracked back to design decisions in early system development process. The systematization provided by requirements engineering has been used to improve the quality of the delivered system documents, models and specification. Nonetheless, in many cases the RE process do not further investigate or verify how the system-to-be will perform based on its design model, relying only on later execution testing.

% both strategically through higher level goals and operationally through lower level goals and tasks

Non-functional dependability requirements are important metrics related to the correct system operation. Dependability defines the ability of delivering a service that can justifiably be trusted. Fail forecasting should provide a qualitative and/or a quantitative evaluation of the system behaviour in respect to fault occurrence or activation, while fault removal includes the verification, diagnosis and correction~[AVIZIENIS]. For some systems, the late fault correction may be very expensive and, in the worse case, may happen after a catastrophic event~[DEPENDABILITY].

In traditional Goal-oriented requirements engineering (GORE) methodologies [GORE CAC], contribution  analysis is based on domain knowledge about the positive, neutral (implicit) or negative impact of a given system alternative to one or more system goals, generally a qualitative goal. By  comparing the overall contribution of two or more alternatives, a decision is made about which one should be adopted for the system-to-be. For instance, if one goal is to communicate with a remote user mobile, alternative means for the notification agent could be to send a SMS, an internet based message or a voice call. These alternatives may contribute with different values for qualitative goals such as `reliable delivery', `fast delivery', `convenient delivery', etc. 

The problem with this approach is threefold. First, it is based on domain knowledge information that may not exist or may not be precise and reliable. As a consequence, the decision of which alternative to use, either at design time or at runtime, may be biased and lead to unexpected violations - the selected alternative may be unable to fulfil its qualitative goals. Second, it is limited to a static representation without any dynamic information that could be used to verify quality attributes that depend on system behaviour and its many nuances such as execution order, cardinality and execution priority. Finally, contribution links are deterministic. As such, the design decision based on contribution analysis is reduced to a simple sum comparison of the concurring alternatives contributions with no support for probabilistic verification.

Additionally to the contribution analysis limitation, the context in which systems operate may not be static. Mobile and pervasive computing, among others, are examples of new computer paradigms for which the environment is dynamic. Battery, signals strength, components availability and the quality of physical resources and relevant information such as the user geographic location may vary through time, posing a new sort of challenge to the development of socio-technical systems based on these paradigms. The contextualization of the informations gathered at RE phase becomes imperative once its validity may be threatened by changing environment conditions~[Finkelstein, CGM]. Accordingly, any improvement of the GORE contribution analysis may have to consider multiple operational contexts.

Static model-based verification is a powerful approach to evaluate metrics over a finite-state model representing the system behaviour. The advantage of this approach is to reduce the correction costs by anticipating failures at early stages of the development cycle. Increased analysis overhead is one of the drawbacks of this approach. Thus, its application must be well justified by the criticality level of the system and by the adoption of an appropriated model-based verification approach.

Despite its valuable contribution to the requirements analysis and engineering, GORE still lacks the proper means to verify the conformance of some important metrics such as dependability requirements and others non-functional requirements (NFR) related to the likelihood and frequency of system execution failures. By tackling the verification of these metrics with a more precise and reliable approach, this work aims to mitigate the occurrence of failures at an early phase of the system development that must justifiably be trusted.

%for these systems must also address the problem of multiple contexts of operation.

%The contextualization of an information means that its validity is not absolute in respect to the state of the world it relates to.

\section{Proposed Solution}

In order to provide a more solid and precise approach for the non-functional verification of different system alternatives and to improve the GORE contribution analysis, we propose the extension of the TROPOS goal-oriented software development methodology with a probabilistic model checking (PMC) approach that has already been explored and is supported by tools such as PRISM model checker[genaína PMC, PRISM]. PMC is a formal method for static verification of system models. System models should represent specified activities that fulfils the root goal or any lower level goal and its elicited alternatives. This model should then be checked for properties such as reliability, availability, performance and power consumption.

The PMC technique requires a behaviour system specification. As the goal model proposed in TROPOS is static, no information regarding achievement/execution order, cardinality and priority of goals/tasks is available, except the activity diagram for the detailing of an agent's single capability behaviour and the sequence diagram for agents interaction. Nonetheless, this problem was tackled by Dalpiaz et al. with a regular expression language to express runtime information, e.g., how many times the same goal should be achieved and the execution order of different system tasks~[RGM]. 

We have used the RGM proposal to fill the gap between the static goal model and its dynamic representation. This dynamic view of the goal model is translated to a probabilistic model following the PMC technique and the PRISM model checker as the tool to automate the verification. PRISM language is used to define the probabilistic model and the Probabilistic Computation Tree Logic (PCTL) to describe the properties to be verified in the model. These properties are derived from the NFR associated to corresponding system goals being verified.

To address the problem of a dynamic context of operation, the context effects over goals, means and metrics should be parametrized to produce a formula that can check the system and its alternatives for different contexts. This verification, performed as part of the Validation \& Verification (VV) phase in RE, should anticipate (contextual) violations of non-functional requirements. 

Treating a detected violation at design time may correspond to actions such as making a different choice for underlying components used by this alternative's tasks, optimizing its behaviour specification or even the disposal of this alternative as a means to satisfy its goal if there is at least one other valid alternative. PMC technique also allows the identification of system alternatives with more influence on each metric through sensitive analysis. 

By coupling a formal verification to a goal model, our approach benefit from a clearer understanding of the system-to-be and its criticality, justifying the verification of system parts or the system whole. The automatic generation of the verification model from an extended version of the goal mode, namely the runtime goal model, greatly reduces the verification overhead. Finally, including context effects provide a more realistic representation of the system to be verified.

Runtime self-adaptation is beyond the scope of this work. However, based on the contextual analysis provided by the CGM and the enriched non-functional and dependability analysis provided by the verification of different alternatives using the PMC technique, it should not be difficult to extend the approach with the additional monitoring, planing and execution capabilities of a self-adaptation loop and have a self-adaptive architecture and mechanism reflected upon its runtime goal model requirements. These concerns should be addressed in future work.

%This space variability problem becomes more complex with the contextualization of goals, means and metrics. According to Bosh et al.[cite Bosh 2004], a high degree of variability allow the use of software in a broader range of contexts. Traditional GORE approaches are mostly used to the selection of which single alternative would exist in the system-to-be for some static context. A dynamic environment can result in contextual violations that can not be resolved just by configuration management. In these cases, a space variability may be required to keep the system properly running in different contexts.


%to a specific kind of goal called \textit{softgoal}. Softgoals are goals for which there is no clear-cut criteria. Often, they represent qualitative intentions of stakeholders, in contrast to the 

\section{Evaluation}

This proposal was evaluated with the application of the extended TROPOS methodology to the development of a Mobile Personal Emergency Response System (MPERS). This system may be seen as a body area network (BAN) with extended functionalities related to ubiquitous emergency response running in a mobile device [BAD]. Instead of a home or hospital static environment, the MPERS is conceived to allow patients with different health risk degrees to maintain mobility while they are monitored and assisted. If a medical emergency is detected, a geolocation feature should point out the location where the emergency response team must be addressed to. 

The MPERS features were based on real emergency response systems available at the industry and also at the BAN explored in previews work [Fernandes]. 

The evaluation process was focused in revealing the major benefits and limitations of the extended TROPOS proposal. Time to market is an important aspect for any software development methodology. Also, the soundness and precision of the proposed probabilistic verification is crucial and must be evaluated as they should not result in mislead decisions about which alternatives should be used by the system. Instead, they must anticipate any violation that could lead to a system failure, specially severe or catastrophic failures, giving analysts valuable information about where the system requirements and specification should be tailored and improved.

\section{Contributions Summary}

This section summarizes the contributions of this proposal.

%A seguir um resumo das con

\begin{enumerate}

\item A new contribution analysis approach for the TROPOS Goal-oriented software development methodology.

\item Inclusion of context effects over goals, means and metrics in the probabilistic model using appropriate constructs and parameters for each case.

\item Conversion rules between different decomposition and runtime constraints in a runtime goal model to a probabilistic model in PRISM language.

\item A parser implementation for the regular expression (regex) language used in runtime goal models with support for execution order, cardinality, alternative execution, optional execution and conditional execution. 

\item An automatic generation of the PRISM model representing activities from a runtime goal model annotated with the runtime regex and graphically modelled using the TAOM4E tool that supports TROPOS methodology.

\end{enumerate}

\section{Document Organization}

This dissertation is organized as follows. Chapter~\ref{ch_baseline} presents the base concepts of this work and the most important related works. Chapter~\ref{ch_problem} details the problem tackled by this proposal. Chapter~\ref{ch_proposal} presents the new extended TROPOS methodology, the rules for the translation between the contextual goal model and the probabilistic verification model, the parser for the runtime regex and finally the implementation approach for the automatic generation of the probabilistic model in PRISM language. Chapter~\ref{ch_evaluation} evaluates the proposal and describes its benefits and limitations. Finally, Chapter~\ref{ch_conclusion} concludes this work with final considerations about the current proposal, related proposals and our future work.
  \chapter{Baseline}\label{ch_baseline}%

\section{Goal-oriented Requirements Engineering}

Goal-oriented requirements engineering brings forward the intentionality behind system requirements. More than just presenting the \textit{what} and the \textit{how} of a system-to-be, it provides the justification for each requirement, that is, they also present the \textit{why}. Through a directed graph tree that begins with a root goal, goals are connected trough decomposition links. Root and higher level goals are related to strategical concerns, while lower level and leaf-goals are related to technical and operational features of the system. 

The main purpose of a goal model is to support the early process of RE, including the elicitation of social needs and dependencies, the actors involved in delivering functionalities and resources, the decomposition of higher-level goals into more granular and detailed requirements chunks, the operationalization through means-end tasks and finally the comparison between different alternatives for the system-to-be. A goal model is said to be valid and complete if it follows all its syntactic rules and if all system goals are either decomposed, delegated to other actors or fulfilled by operational system tasks. 

Three frameworks/methodologies, namely KAOS, i* and TROPOS, represent the foundations for the goal model analysis used by a variety of other proposals [KAOS, i*, TROPOS]. Despite some differences among their syntax, they all share a set of core concepts:
\medskip

\large{\underline{Entities}}

\begin{itemize}

\item \textbf{Actor:} an entity that has goals and can decide autonomously how to achieve them. They represent a physical, social or software agent. E.g.: A patient, an emergency center, a doctor and a Mobile Personal Emergency System running in patient's smartphone.
\medskip

\item \textbf{Goal:} are actors' strategic interests. A goal with a clear-cut criteria for its satisfaction is called a hard goal. In opposition, softgoals has no clear-cut criteria for deciding whether they are satisfied or not and are usually associated to non-functional requirements of an actor. E.g.: vital signs are monitored, emergency is detected, emergency center is notified (hard goals) and system availability, detection precision, emergency awareness (softgoals).
\medskip

\item \textbf{Task:} an operational means to satisfy actors' goals. E.g.: monitor temperature sensor, persist vital signs data, request emergency assistance.

\end{itemize}
\medskip

\large{\underline{Relations}}

\begin{itemize}

\item \textbf{AND/OR Decomposition:} a link that decomposes a goal/task into sub-goals/sub-tasks, meaning that all (at least one) decomposed goal(s)/task(s) must be fulfilled/executed in order to satisfy its parent entity. 
\medskip

\item \textbf{Means-end:} a means to fulfil an actor's goal through the execution of an operational task by the same actor.
\medskip

\item \textbf{Contribution link:} a positive or negative contribution between a given goal/task to a softgoal. Contribution links are used for deciding between alternative goals/tasks at design time (contribution analysis).

\end{itemize}

\section{TROPOS Goal-oriented Software Development Methodology}

TROPOS is a GORE methodology based on the i* framework. Its main improvement is the addition of new phases of requirements engineering and system design, namely~[TROPOS]:

\begin{itemize}

\item Late requirements engineering: Beyond the social dependency modelling with actors diagrams representing stakeholders and their needs in early requirements phase, a late requirements phase focuses on the system actor analysis. In this phase, system goals are inherited from stakeholders needs and represent both functional and non-functional requirements. Each goal has to be further decomposed in more granular sub-goals, delegated to other actors or to be fulfilled by means-end tasks. 
\medskip

\item Architectural design: In this phase, new actors representing sub-systems are created to fulfil different system goals. The idea is to shape the solution using a multi-agent architecture style instead of a monolithic system approach. Data and control interconnections are represented as dependencies.
\medskip

\item Detailed design: The last phase is characterized by the specification of agent capabilities and interactions though UML activity and sequence diagrams. Also, the implementation platform and other specific implementation details are addressed in order to directly map the design to system code.

\end{itemize}

Implementation phase is also specified by TROPOS methodology, but it is out of the scope of this work as our objective is to improve the analysis and the solution that will be later implemented.

\section{Contexts}

Context may be defined as the reification of the environment that surrounds the system operation~[FINKElSTEIN]. Contexts, as already stated, may not be static, but dynamic. A system has no control over its context of operation. Accordingly, a system must be able to support different contexts of operation without violating its goals. Moreover, systems should be able to monitor the state of its surrounding environment and decide which alternative will be used regarding both the availability of that alternative and the optimization of non-functional requirements.

In GORE, dynamic contexts may affect what goals a system have to reach, the means available to meet them and also the quality achieved by each alternative[CGM]. Root goal and higher level strategical goals are not contextualized as they represent the main purpose of a system [Finkelstein]. As these goals are decomposed in more granular sub-goals, a context condition may dictate:

\begin{enumerate}

\item If the goal is required for that context, limiting `what' a system should do;
\medskip

\item If a sub-goal or task is adoptable, limiting the `means' to fulfil a required goal;
\medskip

\item The positive, neutral or negative contribution of using some goal or task to another goal, usually a qualitative softgoal;

\end{enumerate}

The third effect is the main focus of this work, as it is related to the GORE contribution analysis that we aim to improve.
 

\section{Dependability Analysis}

The concept of dependability is related to dependence and trust as well as the ability of a system to avoid failures that are more frequent and more severe than certain threshold~[AVIZIENIS]. According to Avizienis et al., dependability encompasses the following attributes: 

\begin{itemize}

\item Availability: readiness for correct service.
\medskip

\item Reliability: continuity of correct service.
\medskip

\item Integrity: absence of improper system alterations.
\medskip

\item Safety: absence of catastrophic consequences on the user(s) and the environment.
\medskip

\item Maintainability: ability to undergo modifications and repairs.
\medskip

\end{itemize}

Correctness is opposed to failures. A failure is a perceived deviation from system expected behaviour that may have variable degree of consequence on the user(s) and the environment. Failures are caused by a specification faults or specification violations. In the first case, either the goals or the means to fulfil then are incorrect or incomplete. In the second case, system implementation did not followed its operational specification due to a faulty implementation or a deviation from normal behaviour took place in one or more components involved in the execution.

The scope of this work is restricted to specification violations, i.e., we assume that a system specification is valid and has no false assumptions, incompleteness or inconsistencies. Failures are restricted to anomalous behaviour of the technical components or social actors participating in the execution of system tasks. Accordingly, our approach is focused on dependability attributes that should be estimated and compared to their required constraint values and also to the sensitive analysis of how different system parts contribute to the overall value of those attributes. Sensitive analysis may be considered analogous to the original GORE contribution analysis.

%A holistic dependability specification has to include not only the software operation, but also the requirements for which that operation is meant. Requirements are an important factor to decide the acceptable frequency and severity of a software failure. Similarly, context is another factor in that decision. The frequency and the likelihood of failures are related to the dependability attributes of reliability, availability and integrity. Both likelihood and severity of failures are related to safety. 
%
%At early project phases, hazard resolution may involve simply getting more information about hazards or generating alternative design solutions~\cite{Leveson:1995}. In our work, we have used a qualitative means to analyse the dependability to be delivered by goals of a certain system taking into account contextual effects. Our approach improves the understanding of systems fault-causality effect and the identification of best approaches to reduce risk or even determine rates for safety or system level functional failure. Moreover, dependability requirement analysis cannot be accurately fulfilled without taking into account the context under which the system will operate.
%
%
%Avizienis et al \cite{Laprie2004} proposed a failure classification taxonomy with four viewpoints characterizing failures. In our approach, we use two categories: domain and consequence. We use the domain category to distinguish \textit{content} failures from \textit{timing} failures:
%
%\begin{itemize}
%
%\item{\textbf{Content failure}: When the content of the information delivered by a system task deviates from its specification}
%
%\item{\textbf{Timing failure}: When the time of arrival or the duration of the information delivered by some system task deviates from its specification}
%
%\end{itemize}
%
%The consequence of failures enables the definition of failures' severity. Two limiting levels are predefined and other intermediary levels could be defined for each case:
%
%\begin{itemize}
%
%\item{\textbf{Minor failure}: The harmful consequences of failures are limited or at most similar to the benefits provided by the correct operation of the system}
%
%\item{\textbf{Catastrophic failure}: The harmful consequences of failures are incommensurably higher than the benefits provided by correct operation of the system}
%
%\end{itemize}
%
%More details on the complete failure classification can be found in \cite{Laprie2004}. In our approach, a part of this taxonomy is used to guide the definition of the classes of failures severities and therefore the required level of dependability for different system goals. It also takes part in the identification of which dependability attribute is related to each contextual failure occurrence.  

\section{PRISM Probabilistic Model Checker}

A model checking is a formal method that aims to automatically verify if a system model meets its specification for defined properties. Probabilistic and state based model checking supports the verification of finite-state probabilistic models such as discrete-time Markov chain (DTMC), continuous-time Markov chain (CTMC) and Markov decision process (MDP), among others. Different types of properties may be defined to verify a system model and reveal meaningful information for systems audition and validation. 

The PMC technique used in this approach is supported by the PRISM model checker tool~[PRISM]. PRISM allows the modelling and analysis of systems which exhibit random or probabilistic behaviour. The decision of using PRISM as the probabilistic state based model checker was due to the number of successful case studies that have used this tool, indicating its maturity [PRISM CS], and also due to its rich environment that is able to represent different kinds of probabilistic models and their evaluations. 

PRISM is suitable for many different kinds of model evaluations depending on the abstraction level, the type of probabilistic model and the PCTL properties to be analysed. PRISM language offers a rich set of constructs that may represent system modules and components, among others architectural and design configurations. Both qualitative and sensitive analysis are available.

As it will be explained in later sections, goal models may be easily extended with the proper information required for the verification of some important dependability attributes. The objective is to anticipate non-functional dependability violations and to support the decision of which alternatives to use in the system-to-be. A model checking technique should be used for dependability analysis as long as: 

\begin{itemize}

\item A formal system model may be built;
\medskip

\item Properties representing dependability attributes may be defined;
\medskip

\item The analysis overhead is justified, e.g., by its criticality.
\bigskip

\end{itemize}

Finally, PRISM also supports a parametric model verification. That is, instead of providing the final evaluation for a given property, models may use parameters instead of initialized variables and the verification will output a parametric formula whose evaluation will check the model for any valid combination of parameters values.

\section{Mobile Personal Emergency Response System}

The MPERS case study will be further detailed in later sections as the goal models generated by TROPOS methodology are themselves useful for communication purposes. As such, this section will cover some non-functional aspects of this system that justify the use of our formal verification approach.

An emergency response system is a mission-critical system for which failures in achieving its main goals by the time they are required may lead to catastrophic consequences on users, i.e., on patients monitored by the system expecting to be promptly assisted in case of a medical emergency. Accordingly, any stakeholder that wishes to offer a service based on this system will have both ethical and contractual obligations regarding the safety of its product, that is, it must use appropriate means to prevent system failures.

Other dependability attributes such as reliability and availability are metrics over the correctness of system behaviour. MPERS is expected to have a high availability - as it must be ready to respond to an emergency that may happen at any time - and a high reliability - as an incorrect emergency response may lead to death or to costly false-positives. Integrity is a less critical attribute in this case, but must also be addressed as patient privacy may not be violated by disclosing his personal health or geolocation info to unauthorized persons. Maintainability is addressed by the use of a software development methodology and by the ability to update emergency rules remotely at runtime.

The estimation of attributes through PMC technique is limited to those that a probabilistic model may evaluate. Dependability attributes have an abstract definition that must be associated to a concrete and verifiable PCTL property. To demonstrate our approach, we verify the MPERS model for the following attributes:

\begin{itemize}

\item \textbf{Reliability}, represented by the probability of a successful execution of all the activities involved in fulfilling leaf-goals of a certain system alternative. It is also know as the \textit{reachability} as the describes the probability of reaching a final and successful system state. 
\bigskip

\item \textbf{Availability}, represented by the power consumption estimation to maximize the time that the system will remain operational depending only on its battery. This attribute is well related to mobile computing. 
\medskip

\end{itemize}

Reliability verification of an Ambient Assisted Living System also based on body-area networks though PCM technique was explored by Fernandes~[Fernandes, 2012]. Reliability estimation demonstrates an example of non-determinism in the verification model that will result in an non-deterministic evaluation result. Moreover, PRISM cost/reward structure is illustrated by the power consumption estimation. Analogue verification could be used for other attributes. The value obtained by this quantitative analysis must comply with the corresponding non-functional constraint associated to the goal model.  

%It will be up to the analyst and stakeholders to define which type of probabilistic model and which PCTL properties must be analysed for each different system. Dependability attributes may be relevant for any sort of system, but are certainly important for systems with some criticality degree, i.e., for those whose failure could have severe or catastrophic consequences for the user(s) and for the environment.

  \chapter{Related Work}\label{ch_related_work}

\section{Contextual Goal Model}

The Contextual Goal Model (CGM)~[CGM] proposed the contextualization of required goals, adoptable means (goals/tasks) and contribution links values. The main benefit of this work is to enrich the original goal model with the contextualization of entities and relations affected by context variations and to provide a rationale for context analysis. In contrast, the main problem tackled by the current work is the verification of non-functional attributes that requires a more precise and formal approach instead of the existing contribution analysis that is based on analysts direct evaluation of the forward impact between goals/tasks and softgoals. 

In this regard, the CGM provided more realistic and precise contribution analysis contextualized by environment conditions, but did not change the nature of the contribution analysis process. Our work has benefited from the CGM conceptual model and has extended the non-functional GORE analysis with a context-dependent formal verification, i.e., that includes different context effects in the probabilistic model used by the PMC to estimate the values of required non-functional attributes of the system and provide a reliable decision criteria for the selection of concurring alternatives in the goal model before it is implemented.

\section{Awareness Requirements}

Souza et al.~[AwaReq] proposed the Awareness Requirements (AwReq) as a meta-requirement in a goal model, i.e., AwReq specify the success/failure rate and temporal constraints for other requirements in the model, including goals, softgoals, tasks and other AwReqs (*-meta-requirement). The purpose is to enrich the original goal model and provide clear-cut criteria for self-adaptation, as runtime AwReq violations should be addressed by corrective actions. AwReq are formalized by a temporal logic formula, namely the Object Constraints Logic with Temporal Message (OCLtm).

Despite its contribution to the specification of meta-requirements in the goal models, AwReq do not provide an approach to analyse and validate its meta-requirements before system implementation and monitoring. Original GORE contribution analysis could be used to define the impact of a given alternative to some attribute or value composing the AwReq. However, the paper focuses only on attributes that can be monitored by the system at runtime. In contrast, our approach relies on the improved contribution analysis, i.e., the model based verification of attributes through PMC technique that can be performed at design time and provide alternative design decision criteria. Moreover, a similar meta-requirement is used by our approach to define PCTL properties that must be checked by the PMC. These properties are also associated to system goals.

\section{Dependability Contextual Goal Model}

This work has been preceded by another proposal concerning goal-oriented requirements engineering, dependability analysis and dynamic contexts, namely the Dependability Contextual Goal Model (DCGM)~[DCGM]. The contribution was focused on both dependability requirements and estimations based on declarative rules and a variable context of operation.

In DCGM, a failure classification scheme was used to classify the consequence level and domain of failures in achieving system goals. This process lead to the definition of dependability constraints that must be achieved by the means-end tasks used to fulfil leaf-goals in specific contexts of operation, i.e., to the specification of contextual dependability requirements. These requirements inherited the same concept of the AwReq, but instead of being static, they could be associated to a context condition. Another facet of the DCGM is the contextual failure implication, which consisted of a dependability specific GORE contribution analysis supported by Fuzzy Logic to define IF-THEN rules between context conditions and the level of a dependability attribute, e.g., availability and reliability.

The main drawback of this proposal was the lack of scalability, as declarative rules must be provided for different goals, attributes and contexts, proving to be a time consuming task for the analysts. A second problem was the subjectivity of the rules, as they were based in domain knowledge. This problem, as much as in GORE contribution analysis, lead to the idea of coupling a more precise and reliable verification of non-functional requirements of a goal model based on the PMC technique. Still, the idea of a failure classification and the specification of contextual dependability requirements were kept, with the difference that now other  attributes besides dependability ones may be specified.

\section{Runtime Goal Model}

Despite the use of goal models to support the monitoring and adaptation functions at runtime, Dalpiaz et al. argued that these works are `using design artefacts for purposes they are not meant to, i.e., for reasoning about runtime system behaviour'. As such, they proposed a conceptual distinction between the static goal model, named Design Goal Models (DGM), and the Runtime Goal Model (RGM) that extend DGM with `additional state, behavioural and historical information about the fulfilment of goals'~[RGM].

The main purpose of the RGM approach is to provide the proper specification of behaviour information among system goals. RGM defines a class model, while the Instance Goal Model (IGM) provides the instance model that must conform to its class specification. IGM are useful to have an instance representation of the RGM provided by the monitoring of the activities involved in fulfilling system goals. If the monitored IGM violates the RGM, then a corrective action would have to take place. Again, our work has benefited from the conceptual contribution, this time by using the runtime regex language to have a behaviour specification for system goals and use it to generate the probabilistic model. In contrast, our work does not cover instance and monitoring aspects and focuses on the \&V phase of RE to anticipate any violation and for the selection of the most appropriate concurrent alternative elicited for the system.

\section{Formal TROPOS}

The idea behind the formalization of a goal model, as proposed by Formal TROPOS~[FTROPOS], is to provide a verifiable specification of sufficient and/or necessary conditions to create and achieve intentional elements and dependencies in the model and invariants for each element. In addition to these conditions, new \textit{prior-to} links describe the temporal order of intentional elements and cardinality constraints may be added to any link in the model. Finally, Formal TROPOS uses a first-order linear-time temporal logic as a specification language.

The nature of the verification of a goal model with Formal TROPOS specification is different from the PMC used by our work. Formal TROPOS aims to provide the information required for a consistency verification of the model. The verification is not only for the abstract TROPOS syntax, but also to domain specific information of how intentional elements are created and fulfilled in time. Once the model starts to have more elements and relations, its consistency checking becomes non-trivial, justifying the use of a formal specification that can be verified by a model checker tool. 

In our work, PMC technique is used to build a probabilistic model representation of the goal model enriched by dynamic specification (RGM) and enable the verification of properties that depend on how activities in the model are organized in terms of time, cardinality and priority and how each activity contributes to the property being verified. For instance, if power consumption is to be checked, each activity has to be associated to a power consumption unit and the global consumption value is evaluated considering any non-determinism specified in the execution workflow. Thus, even if the Formal TROPOS language allows the specification of dynamic aspects of a goal model, it is tailored for a consistency checking and not for the verification of non-functional requirements of the model, for instance dependability requirements.

\chapter{Proposal}\label{ch_proposal}

In the PMC technique adopted by this proposal, a behavioural specification, usually provided by UML activity and sequence diagrams, are manually converted to a probabilistic model in PRISM language. As a goal model goes from strategical root goal to operational leaf-goals, and each leaf-goal describes a desired state reachable by either a delegation to other actor or by a operational task, then a behaviour specification as proposed by the RGM may be seen as an activity diagram and be used to generate a probabilistic model in PRISM language. This allows the model checking of the corresponding goal model as a set of activities for which temporal and other behaviour aspects are specified by the runtime regex of the RGM.


\begin{itemize}

\item Making a different choice for underlying components: In some cases the replacement of a technical component for another of the same class can improve the quality of how they achieve their goal. For instance,
\medskip

\item Behaviour optimization: The quality may also depend on the pattern used for the activities execution. The specification of a different pattern may eliminate the non-functional violation. 
\medskip

\item Contextualizing the alternative: An alternative may only violate a NFR in specific contexts. In this case, different valid alternatives may be used according to the context of operation.
\medskip

\item Alternative disposal: If the alternative is in absolute violation or if its validity is restricted to contexts that have at least one other valid alternative, this branch can be eliminated from the model.

\end{itemize}

To evaluate the current proposal with the MPERS case study, we have used the a discrete-time Markov chain (DTMC) probabilistic model and focused on the verification of properties related to dependability, i.e., the reachability of the final success state of a set of goal model activities that represents: 

\begin{itemize}

\item if the set is composed of the minimum set of activities that satisfies the root goal: its global reliability;  

\item if the set is composed of the minimum set of activities that satisfies any lower-level goal: its local reliability.

\end{itemize}
  \chapter{Modelling the Problem}\label{ch:motivation}

\section{Motivation}

%According to Lamsweerde, a poor requirements engineering (RE) is the major source of system failures. Lack of user involvement, requirements incompleteness, changing requirements, unrealistic expectations and unclear objectives are common causes~[AXEL].

Goal-oriented requirements engineering (GORE) has gained the attention of both academic and industrial practitioners due to its ability to systematically model the intentionality behind system requirements. More than just presenting the `what' and the `how', goal models also express the `why' of different requirements to exist. Its simple graphical notation allows non-technical stakeholders to take part in the analysis process and have a clear view of the system-to-be. Finally, automated model verification should avoid violations of the requirements specification.

TROPOS is a GORE methodology that also includes architectural and  detailed design phases for the development of socio-technical, multi-agent systems. Socio-technical systems provide and control a wide range of daily used services. Often, these systems are responsible for important and even critical requirements whose failures would cause undesirable or intolerable consequences. This requires developers to take dependability into consideration as a first class requirement.

In TROPOS, as in other GORE frameworks, there is no coupling to any specific verification approach for dependability attributes and other non-functional requirements. Contribution analysis is used for the comparison and selection of alternative design solutions based on how each alternative contribute to one or more system goals, usually qualitative softgoals. This approach, however, is not tailored for measures depending on system behaviour or more complex analysis techniques.

In a previous work and following, we have extended the contextual goal model to tackled the context variation effects on dependability attributes based on declarative fuzzy logic rules~\cite{Mendonca:2014}. From our experience, we have considered that a more scalable and precise approach for the verification of dependability metrics in dynamic contexts was needed. Probabilistic model checking emerged as a potential option to cope a goal-oriented methodology with a formal method for dependability analysis.

To the best of our knowledge, formal methods have not yet being used for a goal-oriented dependability analysis. As a fault forecasting approach, the purpose would be to characterize the dependability and the security of a system and to compare concurring alternative solutions - in this case, goal model alternatives - according to one or more attributes. In specific, the reliability attribute should be the focus of this proposal and the resulting verification model should estimate and verify reliability related metrics as a design-time, manual analysis and specially as part of a runtime automated analysis as part of a self-adaptation loop. Finally, the context effects over goals, means and metrics as described by the CGM should be considered in the analysis.

%In dependability analysis, fault-forecasting aims to derive estimates about measures related to the behaviour of a system in the presence of faults. Dependability benchmark enables the characterization of the dependability and security of a system and the comparison of alternative solutions according to one or more attributes~[AVIZIENIS~37]. This benchmark may be achieved by a probabilistic model checking technique coupled to the TROPOS methodology, enabling the (probabilistic) dependability benchmark of a goal model system representation with additional behaviour specification. 


%MAYBE INTRODUCTION
%An important success factor for any software development methodology is a reduced overhead to the development effort, including domain knowledge and tools to support its process. PMC may significantly reduce the occurrence of system failures, but it also requires additional knowledge about its modelling language and verification steps. A goal model extended by behaviour specification overlaps with the UML activity diagram used by PMC.  Moreover, an automatic model generation for the PMC is desirable to reduce the know-how and effort for the verification of metrics for the system, justifying the implementation of this generator as an extension of a TROPOS modelling tool.

\section{Requirements for the Goal-oriented and Contextual Dependability Analysis}

Based on the identified gap of a probabilistic verification of non-functional requirements in GORE methodologies, we defined the following requirements that must be addressed by our proposal:

\begin{enumerate}[R.1]

\item \textbf{Backward compatibility:} Extended TROPOS runtime regex and contextual notation added to the goal model must not conflict to the existing syntax and semantic of the original TROPOS methodology.

%\item \textbf{Optionality:} The use of the probabilistic model checking as a formal verification approach of runtime goal models as an extended TROPOS phase should be optional and not mandatory.
%\medskip

\item \textbf{Verification scope:} The verification scope may be restricted to a part of the system (local goals) or encompass the whole system (root goal), according to analysts decision.
\medskip

\item \textbf{Model generation:} The probabilistic model representing the activities of runtime goal models should be automatically generated from a runtime goal model created in TROPOS late requirements engineering phase.
\medskip

\item \textbf{Tool integration:} The TAOM4E tool/plugin used for TROPOS modelling and analysis activities should be extended with the runtime regex, context notations and verification model generation.
\medskip

\item \textbf{Static syntax support:} The verification model should be coherent to the AND/OR decomposition of goals/tasks and to the goal-task means-end relation of the original/static goal model syntax.
\medskip

\item \textbf{Dynamic syntax support:} The verification model should be coherent to the goal/tasks achievement/execution order, to cardinality and to alternative, optional and conditional achievement/execution syntaxes in a runtime goal model.
\medskip

\item \textbf{Contextual syntax support:} The verification model should be coherent to the context effects over the activation of goals, the adoptability of sub-goals/tasks and over the individual quality metric of leaf-tasks selected for analysis. 

\end{enumerate}  
  \chapter{Proposal}\label{ch_proposal}

In the PMC technique adopted by this proposal, a behavioural specification, usually provided by UML activity and sequence diagrams, are manually converted to a probabilistic model in PRISM language. As a goal model goes from strategical root goal to operational leaf-goals, and each leaf-goal describes a desired state reachable by either a delegation to other actor or by a operational task, then a behaviour specification as proposed by the RGM may be seen as an activity diagram and be used to generate a probabilistic model in PRISM language. This allows the model checking of the corresponding goal model as a set of activities for which temporal and other behaviour aspects are specified by the runtime regex of the RGM.

\begin{itemize}

\item Making a different choice for underlying components: In some cases the replacement of a technical component for another of the same class can improve the quality of how they achieve their goal. For instance,
\medskip

\item Behaviour optimization: The quality may also depend on the pattern used for the activities execution. The specification of a different pattern may eliminate the non-functional violation. 
\medskip

\item Contextualizing the alternative: An alternative may only violate a NFR in specific contexts. In this case, different valid alternatives may be used according to the context of operation.
\medskip

\item Alternative disposal: If the alternative is in absolute violation or if its validity is restricted to contexts that have at least one other valid alternative, this branch can be eliminated from the model.

\end{itemize}

To evaluate the current proposal with the MPERS case study, we have used the a discrete-time Markov chain (DTMC) probabilistic model and focused on the verification of properties related to dependability, i.e., the reachability of the final success state of a set of goal model activities that represents: 

\begin{itemize}

\item if the set is composed of the minimum set of activities that satisfies the root goal: its global reliability;  

\item if the set is composed of the minimum set of activities that satisfies any lower-level goal: its local reliability.

\end{itemize}

Dependability analysis is used to provide information about different dependability attributes related to system failures. These metrics may be specified as non-functional requirements for isolated system functionalities or for the whole system. Instead of softgoals, we use meta-requirements over functional goals with clear-cut quantitative criteria such as `99.999\%' reliable - a probabilistic value to make it compatible with the PMC estimation results.

To perform the , we focus on dependability related metrics that should be estimated and compared to their required constraint values through quantitative analysis. Sensitive analysis to reveal how different system parts contribute to the overall value of those attributes. Sensitive analysis may be considered analogous to the original GORE contribution analysis.



\section{TROPOS to PMC Code Generation}

As we wanted to automate the code generating process for the verification model, the graphical modelling environment that supports TROPOS methodology and the code generation for multi-agents was extended to also generate probabilistic models for the PMC technique.

To reduce the effort of codifying the verification model, an automated generation of the PRISM probabilistic model was implemented based on an existing open source tool for TROPOS development support named TAOM4E[citation]. TAOM4E provides a graphical environment for goal modelling with TROPOS methodology based on the well known Eclipse Modelling Framework (EMF) and Graphical Editing Framework (GEF). The GORE to PRISM generator was implemented as a Eclipse plugin and integrated to the TAOM4E environment. 

The purpose  of the automated code generation for the probabilistic PRISM model is to optimize the formal verification step by abstracting the PRISM language from the analysts and reduce the overhead and time of the model verification. This should increase the feasibility of adopting the extended TROPOS methodology by keeping analysts with their original responsibility of modelling and analysing the system, its social environment and its different contexts of operation.






In terms of a high level system behaviour, each activity has its own states space including success and failure. Our probabilistic verification approach requires not only a formal specification of the system behaviour, but also metrics related to how individual components involved in system activities will perform in respect to the analysed metric. In reliability verification, each component has an individual probability of successfully performing its functional task. Analysts must obtain these values by consulting their manufacturer, by individually analysing each component reliability based on their behaviour specification until the atomic level or by monitoring these components in a testing or production environment. Further details of how individual metrics may be obtained for the PCM may be found at the literature and are out of the scope of this work.
    
  % inserir demais capítulos
  

  \postextual
  \bibliographystyle{plain}
  \bibliography{bibliografia}

\appendix
  %\chapter{Documentação Original}%
\small\begin{verbatim}
% -*- mode: LaTeX; coding: utf-8; -*-
%%%%%%%%%%%%%%%%%%%%%%%%%%%%%%%%%%%%%%%%%%%%%%%%%%%%%%%%%%%%%%%%%%%%%%%%%%%%%%%
%% File    : unb-cic.cls (LaTeX2e class file)
%% Authors : Flávio Maico Vaz da Costa
%%              (based on previous versions by José Carlos L. Ralha)
%% Version : 0.93
%% Updates : 0.5  [??/11/2004] - Initial release. don't remember the day.
%%         : 0.75 [04/04/2005] - Fixed font problems, UnB logo
%%                               resolution, keywords and palavras-chave
%%                               hyphenation and generation problems,
%%                               and a few other problems.
%%         : 0.8  [08/01/2006] - Corrigido o problema causado por
%%                               bancas com quatro membros. O quarto
%%                               membro agora é OPCIONAL.
%%                               Foi criado um novo comando chamado
%%                               bibliografia. Esse comando tem dois
%%                               argumentos onde o primeiro especifica
%%                               o nome do arquivo de referencias
%%                               bibliograficas e o segundo argumento
%%                               especifica o formato. Como efeito
%%                               colateral, as referências aparecem no
%%                               sumário.
%%         : 0.9 [02/03/2008]  - Reformulação total, com nova estrutura
%%                               de opções, comandos e ambientes, adequação
%%                               do logo da UnB às normas da universidade,
%%                               inúmeras melhorias tipográficas,
%%                               aprimoramento da integração com hyperref,
%%                               melhor tratamento de erros nos comandos,
%%                               documentação e limpeza do código da classe.
%%         : 0.91 [10/05/2008] - Suporte ao XeLaTeX, aprimorado suporte para
%%                               glossaries.sty, novos comandos \capa, \CDU
%%                               e \subtitle, ajustes de margem para opções
%%                               hyperref/impressao.
%%         : 0.92 [26/05/2008] - Melhora do ambiente {definition}, suporte
%%                               a hypcap, novos comandos \fontelogo e
%%                               \slashedzero, suporte [10pt, 11pt, 12pt].
%%                               Corrigido bug de seções de apêndice quando
%%                               usando \hypersetup{bookmarksnumbered=true}.
%%         : 0.93 [09/06/2008] - Correção na contagem de páginas, valores
%%                               load e config para opção hyperref, comandos
%%                               \ifhyperref e \SetTableFigures, melhor
%%                               formatação do quadrado CIP. 
%%
%% Definição de classe para dissertações do departamento de
%% Ciência da Computação (CIC) da Universidade de Brasília.
%%
%%%%%%%%%%%%%%%%%%%%%%%%%%%%%%%%%%%%%%%%%%%%%%%%%%%%%%%%%%%%%%%%%%%%%%%%%%%%%%%
%% PRÉ-REQUISITOS (usando LaTeX ou XeLaTeX):
%%
%% * Uma distribuição LaTeX2e compatível
%%   - em Windows, recomenda-se MiKTeX: http://miktex.org/
%% * Logomarca da UnB (arquivos positivo_cor.eps e contorno_preto.eps)
%%   - disponível na página: http://www.unb.br/unb/marca/index.php
%% * Pacotes obrigatórios
%%   - xkeyval, graphicx, boites, setspace,
%%     geometry, atbegshi, hyperref
%%
%% PRÉ-REQUISITOS (usando XeLaTeX):
%% * Fonte "Arial" instalada no sistema operacional
%% * Pacotes obrigatórios
%%   - fontspec, xltxtra (carregar no próprio documento)
%%
%% PRÉ-REQUISITOS (usando LaTeX):
%% * Fonte ua1 (URW Arial A030, PostScript Type-1)
%%   - disponível no CTAN: http://www.ctan.org/tex-archive/fonts/urw/arial/
%%   - no MiKTeX, basta instalar pacote "arial" via MiKTeX Package Manager
%% * Pacotes obrigatórios
%%   - relsize, fixltx2e
%% * Pacotes opcionais
%%   - MinionPro (fonte proprietária)
%%
%%%%%%%%%%%%%%%%%%%%%%%%%%%%%%%%%%%%%%%%%%%%%%%%%%%%%%%%%%%%%%%%%%%%%%%%%%%%%%%
%% PRINCIPAIS OPÇÕES DISPONÍVEIS:
%%
%% licenciatura - O trabalho é do curso de Licenciatura (Graduação).
%% bacharelado  - O trabalho é do curso de Bacharelado (Graduação).
%% mestrado     - O trabalho é de Mestrado.
%%
%% draft - Gera o documento em modo de rascunho: não exibe as imagens e
%%         modifica o espaçamento entre linhas.
%% final - Gera o documento em modo final, não-rascunho (padrão).
%%
%% impressao       - Gera a versão para impressão, sem hipertexto.
%% hyperref=load   - Carrega e configura o pacote hyperref para gerar
%%                   a versão hipertexto para exibição em tela (padrão).
%% hyperref=config - Configura o pacote hyperref para gerar
%%                   a versão hipertexto para exibição em tela, mas o
%%                   pacote deve ser carregado no próprio documento.
%%                   Leia a seção PROBLEMAS CONHECIDOS para recomendação
%%                   de uso desta opção.
%%
%% 10pt - Define tamanho da fonte do texto.
%% 11pt - Define tamanho da fonte do texto (padrão no modo draft).
%% 12pt - Define tamanho da fonte do texto (padrão no modo final).
%%
%% singlespacing  - Define espaçamento simples entre linhas (padrão no modo
%%                  final).
%% onehalfspacing - Define espaçamento de um e meio entre linhas (padrão no
%%                  modo draft).
%% doublespacing  - Define espaçamento duplo entre linhas.
%% baselineskip   - Utiliza o comando \baselineskip para definir o espaçamento
%%                  de linhas. Se não for infomada, o espaçamento é definido 
%%                  pelo pacote "setspace" (recomendável, ou seja, só use esta
%%                  opção se tiver algum motivo para não usar o setspace).
%%                  Pode ser usada em conjunto com uma das três opções acima.
%%
%% prestyle=<val>  - Especifica o estilo das páginas iniciais (agradecimentos,
%%                   resumo, sumário, etc.). Valores típicos são "plain"
%%                   (padrão) ou "empty".
%% textstyle=<val> - Especifica o estilo das páginas dos elementos textuais e
%%                   pós-textuais. Valores típicos são "plain" (padrão) ou
%%                   "fancy" (este último requer o pacote {fancyhdr}).
%% chapstyle=<val> - Especifica o estilo da primeira página de cada capítulo.
%%                   Valores típicos são "plain" (padrão) ou "empty".
%%
%%%%%%%%%%%%%%%%%%%%%%%%%%%%%%%%%%%%%%%%%%%%%%%%%%%%%%%%%%%%%%%%%%%%%%%%%%%%%%%
%% CRIAÇÃO DA MONOGRAFIA:
%%
%% A folha de rosto e folha de aprovação, por padronização estética,
%% são sempre formatados como "onehalfspacing" independente das opções
%% escolhidas. A seção de Referências é sempre "singlespacing". Citações
%% {quote}, {quotation} e versos {verse} são automaticamente "singlespacing" e
%% com letra menor.
%%
%% Os comandos \textlinf (lining figures) e \texttabf (tabular figures)
%% permitem selecionar, respectivamente, números "maiúsculos" (sim, isso
%% existe!) e números monoespaçados - o que só faz sentido para fontes 
%% que conhenham números maiúsculos x minúsculos e monoespaçados x
%% proporcionais (Minion Pro, Myriad Pro, Didot, Apple Chancery...).
%% Essas opção foi implementada apenas para a fonte Adobe Minion Pro no LaTeX,
%% mas funciona para qualquer fonte que suporte este recurso no XeLaTeX -
%% veja a documentação do pacote {fontspec} para mais informações.
%% Use \textlinf quando os números acompanharem texto em letras maiúsculas ou
%% em versalete (small-caps). Use \texttabf em tabelas ou outros lugares onde
%% for desejável que os números fiquem alinhados em colunas. Como alternativa
%% aos comandos, pode-se usar os ambientes {linfig} e {tabfig}.
%% O comando \SetTableFigures pode ser usado para redefinir um ambiente de
%% modo que os números contidos nele sejam automaticamente formatados
%% maiúsculos e monoespaçados, por exemplo: 
%%
%%   \SetTableFigures{tabular}
%%   \SetTableFigures{tabularx}
%%   \SetTableFigures{longtable}
%%
%% A capa, quando presente, não é contada na numeração das páginas.
%% As páginas iniciais, a partir da folha de rosto, são consideradas na 
%% contagem de páginas, mas não são numeradas.
%% As páginas seguintes (lista de figuras, etc.) por padrão são numeradas com
%% algarismos romanos e aparecem no sumário. Entretanto, se prestyle=empty,
%% não recebem numeração nem aparecem no sumário. 
%% As páginas do conteúdo (da introdução em diante) são numeradas com 
%% algarismos arábicos (13, 14, 15...) e começam a partir de 1.
%%
%% O sumário, por razões óbvias, deve colocado antes de todas as páginas
%% que são mencionadas nele.
%%
%% O trabalho é dividido em elementos pré-textuais, textuais e pós-textuais.
%% Estes elementos (alguns são opcionais, converse com seu orientador)
%% são dispostos preferencialmente na seguinte ordem:
%%    ELEMENTO                COMANDOS MAIS IMPORTANTES
%%       Capa                 \capa
%%    Elementos pré-textuais  \pretextual
%%       Folha de rosto       \folharosto
%%       Ficha catalográfica  \cip (impressa no verso da folha de rosto)
%%       Errata               %deve ser criada como documento avulso
%%       Folha de aprovação   \folhaaprovacao
%%       Dedicatória          \begin{dedicatoria}...\end{dedicatoria}
%%       Agradecimentos       \begin{agradecimentos}...\end{agradecimentos}
%%       Epígrafe             %aluno-poeta, esse recurso não está disponível
%%       Resumo               \begin{resumo}...\end{resumo}
%%       Abstract             \begin{abstract}...\end{abstract}
%%       Lista de Figuras     \figuras
%%       Lista de Tabelas     \tabelas
%%       Lista de Siglas      \printglossary[type=acronym] %usando pacote {glossaries}
%%       Sumário              \sumario
%%    Elementos textuais      \textual
%%       Introdução           \chapter{Introdu\c{c}\~ao} ...
%%       Desenvolvimento      \chapter{Nome do capitulo} \section{Nome da secao} ...
%%       Conclusão            \chapter{Conclus\~ao} ...
%%    Elementos pós-textuais  \postextual
%%       Referências          \bibliographystyle{...} \bibliography %usar BibTeX
%%       Glossário            \printglossary %usando pacote {glossaries}
%%       Apêndices            \apendices \chapter{Nome do apendice} ...
%%       Anexos               \anexos \chapter{Nome do anexo} ...
%%       Índices              \printindex %usando pacote {makeidx}
%%
%% O comando \maketitle equivale a chamar:
%% \pretextual\folharosto\cip\folhaaprovacao
%%
%% A opção "impressao", além de ajustar as margens para melhor
%% encadernação, também faz \maketitle chamar internamente o
%% comando \capa.
%%
%% Caso esteja preparando a versão para visualização em tela
%% (hipertexto), o documento pode ser configurado com o comando
%% \hypersetup a ser chamado no preâmbulo - mais detalhes no
%% manual do pacote hyperref. Algumas opções interessantes,
%% com valores de exemplo, são:
%%   bookmarksnumbered=true,
%%	   (itens das seções exibidas no menu do PDF são numerados)
%%   linktocpage=true,
%%	   (links do sumário nos números de página, não nos nomes de seções)
%%   colorlinks=true,
%%	   (indica links pela cor do texto, não por retângulo ao redor)
%%   linkcolor=red!80!black,
%%	   (cor de links em geral, veja pacote xcolor para opções de cor)
%%   urlcolor=blue,
%%	   (cor de links para URLs - comando \url{http://...})
%%   citecolor=teal,
%%	   (cor de links para referências citadas no texto)
%%   pdfstartview=FitH
%%	   (PDF exibido inicialmente com zoom para largura da página)
%%
%% O comando \listas equivale a chamar:
%% \figuras\tabelas
%%
%% O alinhamento e formatação dos títulos é controlado pelos comandos
%% \pretextual, \textual e \postextual, de modo que em cada parte do trabalho
%% os títulos são apresentados de maneira diferente. Se quiser mudar o
%% alinhamento do título numa das partes do trabalho, logo após a chamada ao
%% comando acima, redefina o comando \chapteralign para aplicar o alinhamento
%% desejado. Se quiser fazer outras mudanças no formato (por exemplo, colocar
%% os títulos em versalete (small-caps), será necessário redefinir o comando
%% \chapterformat (veja exemplo no código da classe).
%%
%% Redefina o comando \fontelogo para mudar o nome da fonte a ser utilizada
%% no logo da UnB, sendo que o padrão é "Arial" (XeLaTeX) ou "ua1" (LaTeX).
%% Deve-se trocar por uma fonte parecida (como a Helvetica, "phv" no LaTeX)
%% apenas se não for possível a instalação da Arial.
%%
%% A classe adicionalmente provê o ambiente {definition}, que exibe
%% definições (conceituais) numeradas e formatadas com borda.
%%
%% Alguns dos comandos a serem chamados no preâmbulo (parte do código antes
%% da chamada a \begin{document}) são:
%% \title[] - Título do trabalho.
%% \subtitle[] - Subtítulo do trabalho (se houver).
%% \palavraschave[] - Lista de palavras-chave em português.
%% \keywords[] - Lista de palavras-chave em inglês.
%%   Os parâmetros opcionais definem como o valor será apresentado na página
%%   da CIP e nas propriedades do PDF, já que estas não podem conter macros
%%   (ex.: \title[Um estudo sobre PI elevado a PI]{Um estudo sobre $\pi^\pi$}).
%% \diamesano - Dia mês e ano.
%% \autor - Nome e sobrenome do autor.
%% \coautor - Nome e sobrenome do co-autor (se houver).
%% \coordenador[a] - Coordenador(a) do curso.
%% \orientador[a] - Orientador(a) do trabalho.
%% \coorientador[a] - Coorientador(a) do trabalho (se houver).
%% \membrobanca - Adiciona membros à banca (além do orientador e coorientador).
%% \CDU - Classificação Decimal Universal (por padrão é 004).
%%   Valores comuns para a área (conforme http://www.udcc.org/) são:
%%    004 Ciência da Computação e Tecnologia
%%    004.2	Arquitetura de computadores
%%    004.3	Hardware
%%    004.4	Software
%%    004.5	Interação homem-máquina
%%    004.6	Dados
%%    004.7	Comunicação de computadores
%%    004.8	Inteligência artificial
%%    004.9	Técnicas de informática orientadas a aplicação
%%
%%%%%%%%%%%%%%%%%%%%%%%%%%%%%%%%%%%%%%%%%%%%%%%%%%%%%%%%%%%%%%%%%%%%%%%%%%%%%%%
%% PROBLEMAS CONHECIDOS:
%%
%% * O pacote hyperref, embora muito útil para gerar um PDF para leitura em
%%   tela, tem inúmeros problemas de compatibilidade com outros pacotes -
%%   alguns só funcionam corretamente se forem carregados antes do hyperref.
%%   A opção [hyperref] (ou seu sinônimo [hyperref=load]) carregam o pacote
%%   na própria classe, o que impossibilita que algum pacote no documento do
%%   aluno seja carregado antes do hyperref. Neste caso, usar a opção
%%   [hyperref=config] junto com o comando \ifhyperref para carregar o
%%   hyperref apenas quando estiver sendo gerada uma versão para leitura em
%%   tela, como no seguinte exemplo:
%%
%%     %%% não carrega nameref com opção [impressao] 
%%     \ifhyperref{\usepackage{nameref}}{}
%%     %%% varioref só funciona se carregado entre nameref e hyperref
%%     \usepackage{varioref}
%%     %%% não carrega hyperref com opção [impressao] 
%%     \ifhyperref{\usepackage{hyperref}}{}
%%
%% * Recomenda-se o uso do XeTeX ou pdfTeX, já que o dvips (+ps2pdf)
%%   não suporta adequadamente links com quebra de linha. Quem deseja usar os
%%   recursos gráficos do PSTricks pode experimentar o pacote TikZ como
%%   alternativa que não depende de diretivas PostScript.
%% * O suporte para o modo matemático ainda não está completo no XeTeX. Se seu
%%   trabalho depende significativamente de fórmulas e equações, use o LaTeX
%%   (ou XeLaTeX com \usepackage[no-math]{fontspec} e uma fonte matemática do
%%   LaTeX que combine com a fonte escolhida para o texto).
%%
%%%%%%%%%%%%%%%%%%%%%%%%%%%%%%%%%%%%%%%%%%%%%%%%%%%%%%%%%%%%%%%%%%%%%%%%%%%%%%%
%% ORIENTAÇÕES TIPOGRÁFICAS:
%%
%% * Atenção aos diferentes tipos de traço:
%%      - O hífen (-) liga prefixos, sufixos, partículas e pronomes oblíquos
%%        (ex.: recordar-se, fazê-lo, co-habitar).
%%      - A meia-risca (--) indica intervalos numéricos e liga palavras
%%        independentes (ex.: páginas 1--5, viagem Londres--Estocolmo, São João
%%        del-Rei--MG).
%%      - O travessão (---) é usado em diálogos, quando o interlocutor muda,
%%        e pode substituir dois pontos, parênteses ou vírgulas em apostos.
%%   Antes e depois de travessão emprega-se espaço, nos demais casos não há 
%%   espaço.
%% * O ponto final marca o fim de uma sentença. Se o ponto for usado com outra
%%   finalidade, como numa abreviatura, usar \ (ou ~ se não quiser permitir 
%%   quebra de linha) para indicar ao TeX que não se trata de fim de sentença
%%   (ex.: Dr.\ Spock).
%% * Embora o padrão das normas ABNT (que, com suas "receitinhas de bolo", são
%%   bastante rudimentares quanto à tipografia) seja espaçamento duplo entre 
%%   linhas, o espaçamento deve ser escolhido de acordo com o tipo e tamanho
%%   das letras.
%%   A regra a ser seguida é: quanto maiores as linhas (quanto mais palavras
%%   por linha), maior deve ser o espaçamento entre elas. Muitas vezes o
%%   espaçamento simples é suficiente numa fonte bem elaborada, mas se o
%%   tamanho médio da linha for superior a 90 caracteres, pode-se aumentar a
%%   legibilidade definindo um modesto incremento de espaçamento (algo entre
%%   \SetSinglespace{1.1} e \SetSinglespace{1.15}).
%% * Como um trabalho acadêmico tem margens relativamente estreitas, para
%%   evitar que as linhas sejam muito longas, em modo final esta classe utiliza
%%   tamanho 12pt.
%%   Deve haver interesse nas opções 10pt ou 11pt apenas se estiver sendo
%%   utilizada uma fonte particularmente grande (por exemplo, que produza uma
%%   média inferior a 80 caracteres por linha a 12pt).
%% * Na tipografia brasileira e portuguesa é costume recuar (indentar) a
%%   primeira linha de cada parágrafo. Como esse recuo serve para destacar
%%   visualmente um parágrafo de seu anterior, o LaTeX não faz o recuo no
%%   primeiro parágrafo sob um título. Quem quiser indentar também o primeiro
%%   parágrafo só para "manter o costume" deve adicionar o pacote {indentfirst}
%%   ao seu documento.
%% * Citações com até três linhas ficam no próprio texto, marcadas por "aspas".
%%   Se o texto original contiver "aspas", trocá-las por 'aspas simples'.
%%   Se a citação tiver mais de três linhas e apenas um parágrafo, usar o
%%   ambiente {quote}. Se a citação tiver mais de um parágrafo, usar
%%   {quotation}. Nestes dois casos, não se deve circundar sua citação por
%%   aspas nem alterar as aspas do texto original.
%%   Se a citação começar com letra maiúscula, o texto imediatamente anterior
%%   deve terminar em dois pontos (:).
%% * Números ordinais em inglês são sucedidos por "st", "nd", "rd" ou "th",
%%   de acordo com o seguinte algoritmo:
%%      Se o número termina em 1 mas não termina em 11, use "st".
%%      Se o número termina em 2 mas não termina em 12, use "nd".
%%      Se o número termina em 3 mas não termina em 13, use "rd".
%%      Em quaisquer outros casos, use "th".
%%   Esse indicador de ordinal deve, preferencialmente, ser colocado como texto
%%   superscrito. Usar 112\textsuperscript{th} e não 112$^{th}$. Se não quiser
%%   se preocupar com essas regras, use o pacote {engord} com o comando
%%   \engordnumber{112}.
%% * Para indicar elipse (reticências), use \textellipsis,
%%   usar ... (ponto ponto ponto) não é a mesma coisa.
%%
%%%%%%%%%%%%%%%%%%%%%%%%%%%%%%%%%%%%%%%%%%%%%%%%%%%%%%%%%%%%%%%%%%%%%%%%%%%%%%%
%% FIM DA DOCUMENTAÇÃO
%%%%%%%%%%%%%%%%%%%%%%%%%%%%%%%%%%%%%%%%%%%%%%%%%%%%%%%%%%%%%%%%%%%%%%%%%%%%%%%
\end{verbatim}

\end{document}
